A colleague of Tom McHugh, Jamie Fenton was the second programmer inaugurated at Dave Nutting Associates (DNA) at the suggestion of Robert Northouse from the University of Wisconsin. Jamie had a far more gung-ho about working as a game developer -- as contrasting with McHugh – being just as prolific as her co-worker in creating games such as Gorf and Robby Roto. This extended to her sheperding of the Bally Professional Arcade console project which created a vibrant indie game development scene in the 1980s.

Jamie’s exuberance emanates from the page as she recounts her career in technology which continues to this day. While the conversation only touches up to 1977, Jamie provides some inner detail on the trials and dangers of arcade video game development and the colorful cast at DNA.

\textcolor{interviewee}{Jamie Fenton:} I know Tom McHugh’s wife was a teacher, school administrator or something. I know David Nutting, I saw him 6-8 years ago, he was still going strong and he was working probably with Tom.

\textcolor{interviewer}{Interviewer:} Oh, was he?

\textcolor{interviewee}{Jamie Fenton:} Yeah. He and Tom kept collaborating. I don’t know a lot about Tom’s history. I’m told that after he left DNA. I just remember him going off and still being somebody we remotely supported. Never really saw him ever again. I remember I used to smoke cigarettes like crazy and he hated it.

Of course both of us worked at UWM in various ways with Richard Northouse, a professor who’s still kicking around. He ran an artificial intelligence lab, it was called RAIL (Robotics Artificial Intelligence Lab). One of the people on the periphery was Jeff Fredricksen, who you’ve probably heard of. Jeff was looking for some engineers to staff up his game company, so Richard set up a deal where he’d make money pimping us out, basically! [Laughs]

He did that for two months at a time, a reasonable arrangement, if you will. After that we went on direct salary with Dave Nutting Associates. That was how we got started. You probably know that too.

\textcolor{interviewer}{Interviewer:} You said you were actually working at RAIL? You weren’t just part of the lab?

\textcolor{interviewee}{Jamie Fenton:} I started in RAIL my freshman year, actually. After year two at RAIL I actually wanted to go work at the computer center because they had a bigger computer and could do more interesting things. So I went to work for the computer center for about a year. It sort of fell apart because it was too much work to do a full time job and to go to school (at that time at RAIL it worked okay). I sort of left Richard to go work for the computer center and Richard was not happy. [Laughs]

Part of the reason I left was Richard Northouse was sort of stuck in the FORTRAN world. He didn’t want to use LISP. If you think about AI in any other form it’s essentially pattern recognition and control system steering, which are both important for engineering but are not as sexy as actually building a robot. [Laughs] Those are some of the things I kind of wanted to do. I actually built a robot for him, that was my freshman design project.

\textcolor{interviewer}{Interviewer:} What did it do?

\textcolor{interviewee}{Jamie Fenton:} It was a base. It had two motorized wheels, two casters. You could drive both motorized wheels in the same direction, go forward, both reverse it, you go backwards. If you drove one the one way and one the other, it would spin in place. We would actually use motors that were usually used in automobiles to run the windows up and down. We would put gears on them to gear them down a little bit and it had plenty of torque.

I could stand on it, and in fact there’s a video of me in one of the Milwaukee newscasts of the day of me standing on that thing and zipping down the hall with the little control box in my hand. You put a dry cell battery on it and you’ve got maybe ten minutes of zipping around on it then you ran the battery all the way down. It’s almost like a segway [Laughs] that I made. I didn’t have the pole but I could stand on it. Maybe more like a skateboard, a better metaphor for it. I could go down the hall with it. Won an award. We didn’t build anything on top of it, we just figured we’d make a better prototype some day. We never really got building a robot organized.

That last thing Northouse wanted me to do was basically something like CP/M, an operating system for a microprocessor computer with a disk drive and that sort of stuff. At that time it seemed like it was a silly project. One of the other things Northouse had, he had a computer from the nose part of a missile. It was of course supposed to guide the missile to it’s target. It looked round like it would fit in the nose cone or something. It had memory, a disk that spun, a little core memory with this disk memory. He wanted to build a compiler or assembler for that. [Laughs] I was really not interested in that project. That’s kind of ridiculous.

He got to see how people did computers back before even core memories were really popular. This would have been late 50’s, early 60’s when that particular weapon was ‘on duty’, if you will. I think I saw some video about that system. I think it was actually a liquid fueled rocket that actually had a countdown, so if you needed to attack Russia you needed to pump oxygen into the thing! [Laughs] Really primitive.

\textcolor{interviewer}{Interviewer:} I wanted to ask, I had seen you mentioned you had played some computer games during your time at the University of Milwaukee. Was that actually in the computer center rather than at RAIL?

\textcolor{interviewee}{Jamie Fenton:} Well actually there was- One person from Digital Equipment corporation came by and they demonstrated the Lunar Lander game. It had computer graphics, would go on the surface of the moon. With the keyboard you could adjust thrust and centering and stuff like this. That was one game that actually had a digital display.

My first semester, my freshman year, I basically snuck into the lab that was used by the PDP programmers. They had a PDP computer - 12-bit, 4096 [bytes of memory] - you could key in a program then it could load a punched tape from an assembler of some type. One of the things you could do with it, you could write a little loop that would read a light pen interrupt signal, you could connect a light pen to a computer so you could actually draw graphics. One visible analog converted it into X, the other converted it into Y.

\textcolor{interviewer}{Interviewer:} Vector style stuff?

\textcolor{interviewee}{Jamie Fenton:} Vector style. The first control would let you turn the beam on and off. I was taking a theater arts class, so I had been working on a light pen program and a program. I had a little program, it would have a little guy dribbling a basketball, so you’d a little basketball bouncing up and down. That’s all it did (it might have had something where you could shoot the basketball too) but what was good with it was I actually rolled the computer out of the lab over to the theater arts class and showed it to everybody. [Laughs] “This is kind of cool!” That would be ‘73, somewhere around then.

\textcolor{interviewer}{Interviewer:} Was it a minicomputer?

\textcolor{interviewee}{Jamie Fenton:} Yes, a minicomputer, a PDP-8E. We did some games with light pens. We did have a Pong-type game going. It was after Atari’s Pong came out so we can’t claim to have invented Pong. You could do a Pong game with it. You’d drive an oscilloscope to a digital-to-analog converter. The cycle rate can maybe do maybe a couple of kilohertz to drive. [Laughs] So Lunar Lander, we had the graphics on this little oscilloscope.

The other thing we could do, we had a lot of text games like Hunt the Wumpus. There were a lot of games where you’d go Dungeons \& Dragons type. It’d say, “Go crawl through door number two.” Then it’d tell you “There was a warm breeze coming with a foul smell.” Whatever the heck it is. You could go play games doing that and that was a very common game genre. Still is, but it was very common then because all you needed was a teletype, you could play them on a teletype. A lot of games, a lot of people. A rich history of games.

When I was in high school, I think I would have been a junior, I wrote a game. The first thing I ever did was a game, it played craps. It just rolled a number. Actually I screwed it up because I did a number between 1 and 36. What I should have done was draw two between 1 and 6, actually match the odds of dice. It was “How much do you want to bet?” If you went broke the loan shark would come around, lend you some money, then want it back at 100% interest. It did a lot of silly things like this. My first computer game ever was doing that.

The other really smart kid in school, Bob Kenny I think was his name, he wrote a game that did tic-tac-toe. It was really complicated, it would do matrix inversion and all this sort of crazy stuff. It turned out he went off to work for DARPA [Laughs]. He probably got to work on every cool research project for many years and went to computer bum world.

\textcolor{interviewer}{Interviewer:} You had very particular interests that kind of led you into what you did. Obviously it was kind of convenient that Jeff had been looking for people, but theater and computers, it kind of mixes together.

\textcolor{interviewee}{Jamie Fenton:} Yeah. I definitely followed my dream in that respect because I was really interested in this ‘multimedia’ (I guess you could call it). Now when I went to DNA, I was of course on pinball machines. I worked on pinball machine programming and we did a couple of pinball machines for various people, also using a 4004 microprocessor which is even smaller than the 8080 that was used in like Gun Fight. Eventually went from 8080 to Z80 and in our case we went from the 4004 to a Fairchild F8.

The first thing I actually shipped was called the Fireball pinball machine. That was popular, though I sort of wanted to do a video game. I actually put together a blackjack game again. Dave Nutting made a deck of cards with the suit and the numbers on them. Actually he did the pieces of the deck of cards so I could put them together, a correction of a hearts or clubs or spades. I had a little animation where times he would deal a card it would actually grow, it would look like it’s being dealt.

I did this, I’m gonna say, six months into my job at DNA, and I did it because I was clamoring to do a video game. We took their game hardware and hooked it into a payout machine so if people wanted to bet money on it, they could get more back than they put in, like a slot machine payout. Technically since we never ran it as a business we never broke any laws (we were kind of worried about it, you know).

Then, of course, Bally Manufacturing had this Mafia association so I was kind of concerned about Mafia. Didn’t want to be part of the mob, that’s kind of scary. Turned out that, of course, was a lot of hocum. You had Bally people who were in the mob but we were not anywhere near being in the mob.

I did meet a character once about a year into working at DNA. Bally had a Las Vegas research center or something, what he had done is he had figured out how to build a card counting machine that he’d put in his shoe. He’d go in and count cards in blackjack. He was bragging about it, I bet a lot of people bragged about this sort of stuff. He was kind of shady. I think nowadays he’d get “Haul your ass off to jail” if they catch you doing that. [Laughs]

\textcolor{interviewer}{Interviewer:} You mentioned the blackjack game. You said that was built on the hardware that had already been built, right?

\textcolor{interviewee}{Jamie Fenton:} Right, so it was the same hardware that was in Gun Fight. I think Gun fight was the 8008, then we thought about going to the 8080. I think Sea Wolf was also the 8008. The first generation was based on the Intel chips and the second generation was based on the Z80 and the custom chipset that went into the Bally Astrocade. That’s what Gorf was done in, that’s what Wizard of Wor was done in, and Robby Roto was done in that scheme as well.

Ms. Gorf was done in a third generation based on the same chipset. The chipset baked in two bits-per-pixel and four bits-per-pixel. It became obsolete once RAM got cheap enough that you could have 16 bits-per-pixel which is what you do for direct color if you have a color map. Once we got to 24 bits-per-pixel, that’s when everything got really wonderful. That’s probably the middle 80s when we could get away with that.

\textcolor{interviewer}{Interviewer:} I was curious, do you recall when you were brought into DNA? An approximate timeframe?

\textcolor{interviewee}{Jamie Fenton:} Oh, gosh. Six months in I think it was in the summer of ‘75…? I’m guessing. There’s a resume out there somewhere with a gameography that has better dates on it. I’m guessing it was about then. I remember being sent in the summer. All kinds of things happened.

One time the burglars got in, stole my calculator, stole a couple of other things. One of the things they found was a copy of Playboy I had. They laid that out like they were looking at it and doing something obscene with it or something. I remember before I came into work, Jeff Fredricksen’s wife who’s a real prudish person saw it and got all mad about it. [Laughs] She went up to me and said, “My husband doesn’t need to look at that sort of thing, if he has me.” [Laughs] 

\textcolor{interviewer}{Interviewer:} Tom told me also that he didn’t think you guys were hired together. He didn’t actually know you were being hired until he saw you there. Is that your recollection?

\textcolor{interviewee}{Jamie Fenton:} Yeah. I think I heard about McHugh, but I don’t remember it being put together too much. It was funny, Jeff really liked me. He was always talking down to other people on the team and talking me up. I thought that was a little weird, but that’s just who he is. That’s what he was like then. [Laughs] I don’t know how he is today. I know he’s over at Apple.

Then he was sort of known for turning on a dime. We would have some project going, he would go “Okay, let’s throw all this away and start doing it this way.” He would do that a couple of times a year. It of course drove people crazy. People were doing PC Board layouts, they’d put some work into it, and they’d throw it all away. I was able to go with the flow better than most people so they appreciated that of me.

They told Tom all this stuff dissing me [Laughs], making sure that we didn’t just form a cartel and go off on our own or something. Tom at one point tried to do that and they caught him, got smacked around for that. I tried asking if I could be a partner in the company and got laughed at. That was why I sort of persisted. Tom always worked remotely.

\textcolor{interviewer}{Interviewer:} So that was Jeff. What do you think of Dave? How was he as a boss?

\textcolor{interviewee}{Jamie Fenton:} Well Dave… All kinds of stories about Dave. He seemed to be the grumpy old man. It took a little while for it to feel like he liked me but once I did the game with blackjack I could kind of tell he was warming up a little.

He threw a party out of his place in Barrington and he of course had Phyllis Nutting. Even more of a high-strung woman than Robin [Fredricksen] was at the time, high society type. High town lady I guess you would call it. She and I got much closer in later years, but she threw a party, I guess it was for her daughter. A coming-out party or something. It was a barbeque.

The thing I did, when I was leaving I managed to ding his Corvette, knock of some of the fiberglass of one of the things. That was like a huge embarrassment. I eventually owned up to it, he took it all in stride, but boy took me a while to feel like I was safe after I did that one! [Laughs] He loved his Corvette. Had the biggest engine you could get and just tear around in it.

Of course, David Nutting was probably bankrupt more times that Donald Trump was.

\textcolor{interviewer}{Interviewer:} [Laughs] I heard that said so many different ways. Tom said the same thing, “Yeah he went bankrupt a couple times…” That’s what he knew about Dave Nutting’s background!

\textcolor{interviewee}{Jamie Fenton:} Yeah, he went bankrupt. It’s kind of an inspiration to me. I sometimes think of him as the old prospector that puts saddles on his mule and goes out one more time. When he went to the DNA thing he actually made enough dough to… Well I wouldn’t say retire, but I don’t think he went bankrupt since. So when I worry about, “Okay, I’m going to do a startup. Am I going to fail?” Well Dave Nutting did it five times! [Laughs] I could only afford to do it once. I never have but I certainly have had my financial misadventures.

\textcolor{interviewer}{Interviewer:} You guys were in a factory to start off, right?

\textcolor{interviewee}{Jamie Fenton:} Yeah. It was actually Milwaukee Coin Industries. Dave used to make the quiz games that you see in airports. It would be 35mm film and would have a question. “Who’s buried in Grant’s tomb?” 1 would be ‘Grant’, 2 would be ‘Harrison’, 3 would be ‘Ford’, 4 would be ‘Nixon’. If you pressed 1 you’d win. What they did is off screen the little code with the right answer. It would give you points if you got the right answer and it would fail you if you got the wrong answer. He put these all over the place and he made some money selling them. They eventually went out of style and he went bankrupt! [Laughs]

[Side Note: The quiz machine referred to here is I.Q. Computer, which was created and sold by Dave’s company before Milwaukee Coin called Nutting Industries.]

He used this building in Milwaukee to run this business out of. It was Milwaukee Coin Industries and there was this other business that was related called Red Baron which would rent video games, video game arcades. Both of those were run out of this place, had a big inventory of games that weren’t being used. So one of the fun things was to get a big extension cord, run around, and try all the games out. A lot of these were old pins, electro-mechanical, and shooting games where you had a rifle.

Another thing that Nutting invented was the shooting game with a rifle. We even made a version of it that had a video screen in it, we called it Road Runner. Of course once Warner Bros. found out about it, they made us change it. So it was called Desert Bird. It was still lots of fun but it never made a big splash.

That was about when Dave Nutting dreamed up the Gorf character which looks sorts of like Cousin Itt, a hairy thing with feet. That was the way it originally was. The reason we called it Gorf was my college nickname was Frog (they called me Froggy). I had frog pictures all over my door, that kind of thing. So I was Froggy, Gorf was Frog spelled backwards, which was always pointed out to me by French people, of course. “I didn’t name it because I wanted to diss French people!” Although, there is some stories about French people…

The driving games, I did a driving game called 280 ZZZap. The newest innovation to this game is we had little dip switch settings on it. You could pick which language to put the titles and comments in. We had French, Spanish, English, Italian, and probably German. You’d throw the switch and it would be English, throw the switch the other way it would be French. They of course got translations of all the different messages they wanted to have.

There was one message that would come up and say how good a driver you were. If you did really well, it’d say “You’re ready for Formula 1” or something like that. If you did really bad, it’d say “Go back to driving school.” You’re that bad! When it came time to translate that into French, “Go back to school” in French was “Retournez à l'école”.

One of the things it had in there was a circumflex, the little thing that hangs off the bottom of the ‘c’ (ç). It’s this thing that’s in French that isn’t in English and I didn’t have it in my font. Rather than invent a circumflex character I just made it a ‘c’. It went into production then the French distributor called up and said, “You’ve made an obscenity!” because taking the circumflex out of the phrase changed the meaning of it from “Go back to school” to “You are a pussy”. Genitalia. [Laughs] We had a really way out insult, so what they had to do is they had to rev the ROM of course to get that out of there. Make sure the ones that went off the first batch that did go to France. [Laughs]

[Side Note: While the author does not speak french, it does not appear that “l’eçole” is actually a french word, let alone an obscenity. It’s unclear what modification this could have been, though the phrase as translated in the game has been guessed at.]

\textcolor{interviewer}{Interviewer:} There’s one other thing in that game. There’s a phrase: Zork. Did you put that in there?

\textcolor{interviewee}{Jamie Fenton:} Yeah, Zork. We used Zork. Where it came from, from my recollection, at the computer center there was a joke going around about the systems request you could make. You know, open a file, close a file, launch process. They were called Executive Requests. We invented an Executive Request that was called “zork\$” and it did all sorts of silly things. It was an absurd instruction.

You know Halt and Catch Fire? A bunch of roman numerals, those sorts of instructions. Well Zork was one of these things. We made all kinds of goofy things that it did. When it was time to go ‘Bang’ or ‘Zap’ or something, rather than use that we had a ‘Zork’.

Of course we used the name Zork and a big adventure game started using Zork. I think they probably own the trademark for it. We weren’t influenced by the Zork people, we were influenced by our culture at UWM.

\textcolor{interviewer}{Interviewer:} Yeah, yours definitely came before which is why I was kind of curious about it.

[Side Note: The name “Zork” was used by early MIT hackers and eventually assigned as the name to the game we have come to know. It is possible that the phrase developed relatively early on at either the MIT or Milwaukee computer labs; saying that it pre-dated the use as the name of the game “Zork” doesn’t mean to imply providence.]

\textcolor{interviewee}{Jamie Fenton:} Yeah. Maybe Zork was around then back in the early 70’s and that’s where it bled over and copied into us. For us it was a joke, a systems request on the UNIVAC 1108 was what Zork was. I don’t think anyone’s litigated about it, mostly because that was so early in the era that the copyright law wasn’t really written yet. I guess they could use the trademark laws to stop Road Runner. Probably could trademark a name and then you couldn’t use it in a game and it wasn’t a problem. Might not even be a problem now, but I don’t know. That’s what lawyers thrash out.

Jeff had a lawyer that he got (trying to remember his name) but he was kind of weird. A little shady, was also in the Air Force. Him and his girlfriend tried to befriend me. They were always trying to clean me up and turn me into a respectable person. [Laughs] Sometimes they would do stuff and if I did it again next time they’d get offended. A little bit weird.

\textcolor{interviewer}{Interviewer:} You said when you first got in you were doing the pinball. Dave and Jeff had done that version of the pinball already. I forget what game they hooked it up to… Flicker.

\textcolor{interviewee}{Jamie Fenton:} Yeah, they did Mirco. There was Mirco Games [in Arizona] that they built while they got their patent filing underway. Jeff built it, it was pretty simple. You can kind of credit him as the first programmer of both video and pinball machines. The video game was very simple, I think it just made a character walk back and forth controlled by the handle. I never saw it, but that’s what they demonstrated to Bally/Midway.

The pinball machine they did the same thing with, but they also had Mirco, which is a pinball company, as an alternate client so they could sell to either of them. They were basically doing a smart move. Always have buyers and sellers for anything you need or want. [Laughs] Then you get a good deal! They were doing that.

[Side Note: Dave Nutting and Jeff Ferdricksen’s first project with DNA was creating a prototype microprocessor pinball machine for Bally, as well as patenting the concept. Bally passed on producing a machine based off the design, so Nutting sold his concept to Micro Games in Arizona who produced the Spirit of ‘76 pinball table, the first commercially released game of that type with a computer and software to run it.]

I don’t know who else they approached. They may have approached Atari… Atari didn’t exist then. I know Bushnell approached Bally and got turned away with Pong. One of those classic stupid moves. [Laughs] It’s like the guy if he had kept his investment in Apple he would have \$23 billion today, that kind of thing.

So the pinball machine, Jeff had a program that I could refer to see how he accomplished that sort of thing. We figured out to make a little virtual machine that would make it so you didn’t have to explicitly script stuff. You could actually whip out a game really fast. There’s a story of Dave Nutting giving me the specs for a second game and he thought it was gonna take six weeks. He was surprised I had it done that afternoon. [Laughs] I had this little virtual machine and it worked out okay. All I had to do was just code that out.

\textcolor{interviewer}{Interviewer:} Did you actually work on the Mirco pinball machine?

\textcolor{interviewee}{Jamie Fenton:} I think I did, but it didn’t get very far. I did work on the Mirco for a bit. I think we sort of had the same hardware in mind. I can’t remember if Mirco used an F8 or if they used the 4000.

\textcolor{interviewer}{Interviewer:} They used the F8. 

[Side Note: The Spirit of ‘76 pinball machine actually used a Motorola 6800. The error is on the author.]

\textcolor{interviewee}{Jamie Fenton:} If they used the F8 then I worked on it, because I was the only person who did F8 at Bally/Midway. Not even the engineers. If it was F8, it was me.

\textcolor{interviewer}{Interviewer:} How did you go about learning some of this stuff? Mainly it was the kits rather than the programming language that was the problem. Tell me a bit about the development process of some of these games.

\textcolor{interviewee}{Jamie Fenton:} Well, one of the things Jeff did that was a little bit controversial is went out and bought a MODCOMP minicomputer. MODCOMP made a computer that was actually a pretty good minicomputer. There weren’t a lot of them made, the big customer for these things was NASA  up until they shut down the space shuttle program. Even though it was big and heavy, it worked! They kept using these things.

We could buy what was called a ‘cross product’. It was an assembler, it was written in FORTRAN. It could be put on a minicomputer, you run it, it’d spit out some hex, then we could go to the microcomputer. As soon as possible, we got it so we could edit text on the microcomputer and assemble it on the microcomputer, but for the first year or so we used this minicomputer. As soon as we figured out how to do it locally this minicomputer was sitting around, collecting dust, because nobody need it anymore. They were still paying off on the lease for five years and they couldn’t ditch it. It was kind of a stupid move. That helped initially get things assembled.

In terms of working with the boards, we did ‘bring-ups’ which people still do today. Usually the first bring-up is the simplest program, almost like ‘null job’ or something. It’s maybe 20 instructions long, it runs through, makes a port increment, maybe changes a bit on the screen back and forth. Just something to, say, ‘hang the line’. Work with that to get the board debugged enough to run with the microprocessor. It’s sort of a bring-up process which is something people still do today, it’s the same thing. It’s usually the first of any system. Android, iOS, or anything. It would be kind of painful and indirect until we got the APIs nailed down and figure out that the system is going to work, and so forth. That wasn’t changed.

The oscilloscope was our friend. There were logic analyzers but they were super expensive. We did start using them later, but then we’d just get the oscilloscope out. If you could just get the oscilloscope to make a square wave coming out of the I/O ports, that was like the first sign of life and your thing was working right.

One thing, of course, that went into these games was a built-in test. There was one that would run that was more expensive and it would run if you did the ‘text switch’ which was a dip switch bit. It would go through a comprehensive memory test, it would set patterns in the memory to make sure the memory’s good, and it could do a couple of other things. If you were on a pinball machine with wires and solenoids, it played each of the lights separately so you can check to make sure that all those things work and no shorts.

Those would go in the product, but simple inversions of the same idea would go into the burn-up process. You would assemble something really simple or even in the crudest case punch in a tape where the hex would be hand assembled. To get started, you would work from there or upwards.

What you can do, you can burn ROMs (EPROMs) as sort of a stage. We bought tons and tons of EPROMs. Burn EPROMs from the machine that was designed to do that which had one of the machines attached to it. You’d stick the EPROM in, reboot it, and see if it ran. That was sort of the step up. If you could ever get to the point where you could run an in-circuit emulator, which is a device you could run in place of a microprocessor-

\textcolor{interviewer}{Interviewer:} Yeah, Tom said it was called the ICEbox, as you called it.

\textcolor{interviewee}{Jamie Fenton:} Yeah, the ICEbox. There were earlier ones when we were using the earlier systems to do the same thing. The in-circuit emulator, that was much, much easier because you could generate, test, compile, test, compile, test many times a day rather than burning an EPROM was like half an hour turn around.

Moreover, EPROMs used ultraviolet light. I remember I looked and ended up getting my eyes burnt by ultraviolet light. Boy, that hurt like hell. It was one of the most painful things you could ever imagine. Thank god my eyes healed. [Laughs] You have to have a lot of respect for the ultraviolet light that coated EPROMs. So I used the ICE, I did other variations and so forth, onward and onward.

\textcolor{interviewer}{Interviewer:} In the early days, the 1975 period, who were the people who were actually there? It was you, Jeff, Tom. Was Dave Otto around then?

\textcolor{interviewee}{Jamie Fenton:} Dave Otto was there. Dave Otto was Jeff’s ‘gopher’, if you will. Jeff needed an assistant and he’d go run around run things. You probably know Adam McNeil came up with Evil Otto in one of his games which was based on David Otto. Otto at one point spent a couple of months as a reserve cop somewhere. He really wanted to be a policeman and he even had lights on his car, but he wasn’t actually a law enforcement officer. I used to call him a closet cop. [Laughs] Not to his face, but David Nutting thought it was pretty funny!

I got along with Otto okay but he was kind of awkward in his own way. He told me he was a virgin, which I believed. [Laughs] He sort of turned into the office manager and once he was office manager his next decree was everyone would show up at 9 in the morning and work ‘til 5 in the afternoon. As I like to put it, I never once complied with those instructions!

\textcolor{interviewer}{Interviewer:} [Laughs] I can imagine! In a place like that? No way.

\textcolor{interviewee}{Jamie Fenton:} Yeah. He didn’t take it too far and had him cut down a peg or so, but he still did a lot of gruntwork of keeping DNA going. I think he got to the point he did some petty stuff. 

There’s somebody that Jeff had that laid out PC boards, can’t remember his name, but he was just faster than all hell. He could do a PC board in a day. Some of the later people were a couple of days and it would take three months for other people to do. He was just a crackerjack PC board guy. I can’t remember who he was but he was astonishingly good at doing PC boards.

Of course we used a lot of wire-wrap prototyping. One of the things they had me do was a wrote a program called ‘Wire Wrap Wizard’. What it would do, it would print out a punch list for somebody to do the wire-wrapping with. Every time it would say, “connect this x,y to this x,y”. It was designed that, for example, if you want to connect two bottoms together then relay the two tops you only have to rip one thing out you don’t have to rip all the wafers and sketch codes.

Typically there’s a wire-wrap and it’s got maybe three levels. The first level’s right up against the base of the board, so you want to connect from one of the pins at the bottom to the next base at the bottom. Then from the next one you have to go to the next level up, you want to run that over to that level, not down to the lower levels. What that meant was, you did everything in odd and even order. 

By doing it that way, if you had to undo something you didn’t have to rip everything out. At worst you might have to take out three of them, you wouldn’t have to take out the whole run like VCC [voltage common collector] or something. VCC was a place where you could get that you could get grounded VCC just the pins on the PC boards. You could do those.

I wrote Wire Wizard and it ran on the MODCOMP, so after a while it was like the only thing we used the MODCOMP computer for. You input the circuit list into it, you could trace along and say “Connect this pin on this part to this pin on that part” and so forth. It would call descriptions of chip number and pin number to the wire wrap D14. [Laughs] It did all this. It also tried to figure out how to do it so it would use the least amount of wires. That’s what we prototyped with.

I wish I could remember the name of that engineer that was just so crackerjack. Some of the people we had later were much slower. Of course, the boards were more complicated too, but it’s not quite the same. I think there’s another person who’s trans now and he was working on boards but this is back when we were in Chicago, not Milwaukee. Back then we were both in ‘boy mode’ if you will. [Laughs] ‘Guy mode’ as we call it.

\textcolor{interviewer}{Interviewer:} Did you and Tom ever worked together on a project? He recalls doing all the programming on everything that he did, but I know he also said he wrote the Midnite Racer/280 ZZZap game which I know you said you did.

\textcolor{interviewee}{Jamie Fenton:} Yeah, that one… There were two versions of the ZZZap game. One was for the Bally Astrocade and the other one was done by Bally/Midway. Somebody else wrote a different version of the game for a different processor that we somehow god the source code to-

\textcolor{interviewer}{Interviewer:} Does the name Ted Michon ring a bell?

[Side Note: Ted Michon created a version of the game Nürburgring for the company Digital Games. He claims to have created a microprocessor version first and then shared the code with Dave Nutting Associates.]

\textcolor{interviewee}{Jamie Fenton:} That might be. What I remember is, he had a table that did one over z to figure out the projection of a road. He did a table that could calculate that so you didn’t have to do divisions, but I set up a logic for how to do the game that had French installed in it. That was definitely me who did 280 ZZZap. We never really worked much together.

We have have done a tiny little bit. I remember the virtual machine I did, Jeff asked if he could use it in one of the Sea Wolf games. He may have taken some of the ideas I had and used them but we never pair programmed on anything. That was so far from being possible you could not imagine. 

Pretty much the way the mind was then, the programmer owned his program and you didn’t collaborate on it unless you really, really, really were forced to. Something like an internal revenue server some place, something that’s way more complicated than any human being could do. If you could fit it into one human being, that was the way it went.

\textcolor{interviewer}{Interviewer:} Though both Jeff and Dave had some kind of programming knowledge, right?

\textcolor{interviewee}{Jamie Fenton:} Well Jeff had some, Dave didn’t really. He knew enough about it to understand what we were talking about, but he certainly never did any programming to my knowledge. What he would do is he would draw out the characters on graph paper and I would have to know an available template that marked off the byte boundaries. I put that up against the graph and remap the hex that I was supposed to put in the assembler so that we could get the character in there. That was the first art pipeline, was doing that. Eventually we would get a woman with a paint program to make art, but back then I had to digitize graph paper. That’s how we did that.

So David pretty much deferred to Jeff on the technical issues. It was all about the industrial design of the games.

\textcolor{interviewer}{Interviewer:} Did he actually build the cabinets?

\textcolor{interviewee}{Jamie Fenton:} Yeah, he would build the cabinets, the prototype cabinets. Sometimes the games had reflectors in them and he designed the reflective system. You could have the monitor facing a weird direction, look in it, mirrored it, make sure the reflection was in a different place. He did that sort of thing, but he never programmed. He probably could have never learned. Maybe he did eventually, but a lot of people sort of thought it was easy to program or you had the aptitude for it you were sort of a genius, or you weren’t. He probably wasn’t.

I thought Jeff, for somebody who was just hacking it out of whole cloth and just learning, did really really well. I think highly of what he did but he was learning as he was going. Jeff’s thing is he was a radar operator in the Air Force and then he was going to become a television repair man. Then this came up and he figured out how the bet this. He bootstrapped at a career now.

One of the things I’m doing with Mom here is go through all the boxes, photos, and papers and stuff from our past. It’s been helping her keep her memory going, so it’s a good thing. I tell ya, when you get into your 60s and you’re trying to find work in Silicon Valley, all the humiliations you have to put up with! [Laughs]

\textcolor{interviewer}{Interviewer:} If you do find anything in that old stuff, that would be very interesting.

\textcolor{interviewee}{Jamie Fenton:} Rattling around in the back of my car I’ve got a box of listings but they were sort of more from the DeFanti era. What happened is, after I moved to Chicago I went to California. In California I bought a book called Computer Lib/Dream Machines by Ted Nelson which I read on the way home. It was like really cool and turned me into a personal computer revolutionist. Then I of course tried to get Jeff to believe in all this stuff.

One of the things he went through a lot of discussion with the book was about Tom DeFanti. I actually found Tom DeFanti and started taking his class. I introduced him to Jeff, and then Jeff, Tom, and DNA had a relationship for years. I eventually wound up moving in with him, living in his house for a while which Ted Nelson used to live in as well. Eventually I bought the place and lived there until the time I went out to California with Macromind.

Tom DeFanti and I go way back. I have all kinds of stories, strange directions, and outcomes. There was a time when I was kind of a rebellious rival of his and looked down on him a little bit because I was too full of myself. Of course nowadays I looked back and think what a jerk I was! [Laughs] 

\textcolor{interviewer}{Interviewer:} Best of luck with your career! Thanks again for your time.
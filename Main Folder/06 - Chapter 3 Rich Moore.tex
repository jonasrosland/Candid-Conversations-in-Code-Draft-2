Representing a new wave of programmers which entered the industry in the late 1970s, Rich Moore was a guru who had his computer education enhanced by the presence of popular mainframe games. Most notably, this would prime him as the programmer for Lunar Lander which was in part drawn from memory rather than strict recreation. He would also start the process of 3D game development at the company with the flight game Red Baron, also serving as the technology base for Battlezone.

Rich has a large deal of insight about the culture of Atari as a whole, vividly describing the personalities he worked with as the hard-working savants that they are. After leading a few games on the technical end, Mr. Moore would move into a management position in the arcade division, which provided the programmers with a knowledgeable person in the larger decision-making process. Rich Moore describes how he got to that stage by the early 1980s and the exciting game development stories from the world’s greatest arcade company.

\noindent\makebox[\linewidth]{\rule{\paperwidth}{0.4pt}}

\textcolor{interviewer}{Interviewer:} Like a lot of the Atari employees were you from California?

Rich Moore: Yes. Born and raised in California.

\textcolor{interviewer}{Interviewer:} What area?

Rich Moore: Born and raised from San Jose. My dad actually worked at Lockheed. Actually born in Berkeley when my dad was going to school there; I went to school there as well. It turns out where my wife and my daughter went to school too. So I have pretty strong ties in the Bay Area in total. It's kind of the Silicon Valley/San Jose area as well as Berkeley. The Berkeley, Oakland, San Francisco area.

So yeah pretty familiar with the culture or the environment of Silicon Valley. I’ve lived here a long time in the many decades so saw all the home development, obviously business and commercial, but also the teardown of orchards that used to predominate the Silicon Valley area

\textcolor{interviewer}{Interviewer:} Was your father an engineer for Lockheed?

\textcolor{interviewee}{Rich Moore:} Yes he was. Not a programmer, he was an electrical engineer. He worked on the space program. So tech was a pretty active part of growing up as a kid. I actually had one of the first programming courses - although I don’t know if it was really formally a programming course - in high school. For lack of a better word I guess: the gene pool or gene source is in the tech tree.

That's interesting to me because now because all the people that are in entertainment or gaming say, “Oh yeah my first thing I did, I remember having an Atari 800 or 400, 5200, 2600 and that’s how I got interested,” in gaming or tech or whatever. Everytime I share a story, either with a new client or new work or team or whatever, and you share “What do you? What do you know?” You mention Atari, you see people’s faces just light up.

\textcolor{interviewer}{Interviewer:} When you did that introductory programmer course what were you programming on?

\textcolor{interviewee}{Rich Moore:} [Laughs] It was a Hewlett Packard. The best I would describe it as a programmable office desktop calculator. These predate the HP-35 and 45 handheld calculators, a larger format of that. It had a card reader, but the card reader you actually filled them in with number two leaded pencil, you filled in the little squares. That’s how you coded and programmed. Essentially you were doing assembly language for this calculator. It was kind of a smart calculator, which at the time was really a small computer. That’s how it got started. Not punched cards, but the same physical format.

Interview: At Berkeley did you go for programming? Was that a thing that you could do at that time?

\textcolor{interviewee}{Rich Moore:} Yeah. It started in the early 70s. It had just moved from a science school to a formal engineering school or School of Engineering there. It was formerly WCS. Specializing in the computer science degree is what I did. People could also just do Comp Sci directly and that was typically on the science side. Probably in no small part because of my dad's background, I had a stronger interest in sciences, physics and stuff like that so I want to go to engineering school more than just science.

I did everything there. Started with punch cards on CDC [Control Data Corporation] mainframes: that only lasted one year. That was converting at that point in school from punch cards and Hollerith codes very very quickly to Digital [Equipment Corporation] mainframes and mini-computers, and then languages. So it was rapidly transferring to Pascal, C, LISP, Fortran, all of those languages I had experience with and have utilize that at school there.

Interview: What sort of applications were you being taught to create? Obviously not games!

\textcolor{interviewee}{Rich Moore:} No, it wasn’t games, but it’s interesting you bring that up. The first game I programmed - well, second game, I did a pinball and a couple of other small projects - was Lunar Lander which was the first vector-generator (XY) game that Atari manufactured and produced. I actually had a job on campus in the Electronics Research Lab and they had a Hewlett-Packard mainframe or minicomputer there, and they had a display and the game they had on it was actually Lunar Lander. [Laughs] That was pretty appealing. I was pretty fortunate and pretty lucky. That was in the labs, it was not accessible by very many students there.

HP did that as a bit of a demonstration of that hardware, like today you have your pre-load applications on whatever you bought, right? That was kind of one of the pre-load applications to show you what was possible with these computers to get you interested in their hardware. So we actually had a Lunar Lander. It was pretty fortunate, my first boss there, a new project happened to open up for the games. You know, “Do you happen to know about the Lunar Lander game?” “Uh, yes! I do!” That was fortuitous because I could relay on exactly how the gameplay was and solve the nature of the design. I spent… More hours than I should have been playing it there.

[Side Note: The Lunar Lander program that Mr. Moore would have played was originally created as a display for a DEC terminal, rather than Hewlett Packard. The program – created by contractor Jack Burness – may have been ported to other terminal systems, or Rich could be misremembering.]

Interview: [Laughs] I was curious, I've seen some photos of the Lunar Lander game where they're playing it with a light pen. Is that how you played it?

\textcolor{interviewee}{Rich Moore:} Yep, exactly. I’m trying to think if there was any other control interface was besides that… I’m sure there were other things that were on there, more function and business applications on there as well. That’s obviously the one that stuck in my head, obviously the Lunar Lander game.

Interview: So what what kind of programming were you looking to go into then?

\textcolor{interviewee}{Rich Moore:} So that's a great great question. I actually I took a little bit longer - I took like four years and two quarters - to graduate and part of the reason I took that I took a little bit more is I actually did a co-op work - a work-study job - down at Hughes Aircraft in Gelndale, Southern California. The project I had there was actually microprocessor based. It was the radar group and they had a bunch of different test systems. This was the first one that was a test system for the power supply to the radar system that was being automated and being used in programmable testing. So I got a chance to work with microprocessors processors and doing assembly language there, and probably one of the first jobs there doing that as a co-op. [Laughs] I was not even a full-time person there doing it.

[Side Note: Berkeley had a work study program for enterprising students to take classes for six months a year then work for real companies as interns for six months. Both Steve Bristow and Al Alcorn of Atari took this course to work at Ampex Corporation where they met Nolan Bushnell and Ted Dabney.]

That actually led to Atari. I worked there for six months made some nice money. There was a fair amount of overtime that was required cuz they're trying to get done during a timeframe. So it worked out pretty well! Funded the rest of my school and stuff after that, and that microprocessor experience turned out to be very valuable. The difference in interviewing afterwards, the confidence and everything was just completely different. The work environment and having that history was a pretty pretty big deal.

I didn't interview with Atari on campus, but the reason I got selected and was able to interview with Atari at work was that my boss scanned resumes. We did all that stuff at the job placement center at Cal back then. My boss scanned resumes and he saw that I had microprocessor experience. Atari at that point was going from dedicated hardware to microprocessors - the 6502 back then - for their game boards.

So again: Fortuitous, worked at the lab on campus; fortuitous, knowing Lunar Lander; fortuitous, having done the work study at Hughes and then working on microprocessors. It all lined up, like dominoes.

\textcolor{interviewer}{Interviewer:} What was the microprocessor you were using at Hughes?

\textcolor{interviewee}{Rich Moore:} There were two primary ones we were using at Atari. We were using the Motorola 6800, which was Motorola, and we were using the 6502. Can’t think of the manufacturer…

\textcolor{interviewer}{Interviewer:} MOS Technology.

\textcolor{interviewee}{Rich Moore:} There you go. Both 8-bit processors, pretty rudimentary especially compared to now. Pretty basic. Those two processors were the ones we used. Before too long, probably matter of maybe two, years we'd upgraded. I think it was the 6809 and then the 68000 which was the first 16-bit microprocessor that came out. All the systems kind of updated as the technology improved.

\textcolor{interviewer}{Interviewer:} So were you doing 6800 at Hughes?

\textcolor{interviewee}{Rich Moore:} I was actually doing the Intel 8080 at Hughes Aircraft. The concepts were all very similar across the different families. It’s really just, what's the number of registers. It was pretty much all assembly language at that point. We did upgrade to C probably in that two-to-three year time horizon, but at that time I think it was all assembly. 

It was relatively easy. You had the little programmer card, a manila card board. They kind of fold it back and forth vertically and had all the instructions on it. That made it pretty easy to move back and forth across stuff.

\textcolor{interviewer}{Interviewer:} I saw that a Rich Moore at Atari was mentioned as a pilot at some point. Were you a pilot back then?

\textcolor{interviewee}{Rich Moore:} I was not a pilot. There were a couple Rich/Ricks there [at Atari]. There’s a fellow named Rick Moncrief you’ve probably run across. He was a pilot, Dave Sheppard was a pilot, Dave Stubben who was VP of Engineering was also a pilot. So there was a few people who did fly. I did - after Lunar Lander- Red Baron which was a flight game. Part of the prep for that game, Rick Moncrief - I don’t know if he owned it at the time or had a joint-share ownership - a small airplane and we went on a flight. Took us up in the air around the bay for a couple hours. The whole idea is to kind of get a sense of what's some of the characteristics, what would feel like to fly and stuff. To utilize that in the game to get a sense of not only flying around but what's some of that physical attributes of turning left andturning right.

\textcolor{interviewer}{Interviewer:} I guess this was a Rich Moore in the computer division.

\textcolor{interviewee}{Rich Moore:} Yeah, it’s not an uncommon name. There’s the Rich Moore director that did Wreck it Ralph.

\textcolor{interviewer}{Interviewer:} [Laughs] Yeah!

\textcolor{interviewee}{Rich Moore:} I didn’t necessarily know everyone on the consumer side. I knew quite a few of the people but not everyone. There’s are a number of Ricks and Richs, and there are some names that are really really close.

\textcolor{interviewer}{Interviewer:} So, you told me a bit about how you got into Atari coin-op in the pinball division. Tell me a bit about what was your impression when you first got there.

\textcolor{interviewee}{Rich Moore:} Ahhh, it was fairly casual. I never felt it was high-stress. Most of the people there, I mean I was young out of college, most of the staff was young. There was a team attitude not only working at the same company and being part of the same organizational team, but also just since the lifestyles or age and stuff like were very similar, going through the same kind of stuff. It was almost like our lives were marching kind of in sync. People getting married people, on their first kid, people buying their first car, buying their house. We were mostly just out of school so kind exploring going from the cheap beers and the cheap wine to a little higher class beer and wine! [Laughs] I started to discover cocktails. We spent a lot of time both in the office at work but also a fair amount of time socializing outside of work.

Then as you got into it there were a number of different trade shows that happened. So if you were successful and kind of built up your career there you generally would rotate in and be able to go to the trade shows. You could help build up the community because you're not only working at your job but then as you finish your game you got the opportunity to kind of demonstrate it and show it off to distributors and operators (for me in coin-op). For the consumer side it would be like CES shows.

I went beyond - even though my role was developer and a coder - I still got exposure to the overall industry. That whole dynamic. By industry I mean marketing, sales, and the business aspect of “Okay, what are you gonna price this at? What's the competitive landscape of other games out there at the same time? What's the kind of dynamic of manufacturing? What's the release quantity? What’s the re-order?” Stuff like that.

\textcolor{interviewer}{Interviewer:} So you were you were paying attention to that kind of stuff even though that wasn't your specific area when you were first brought in.

\textcolor{interviewee}{Rich Moore:} Exactly. Everyone’s ego was - people had pretty strong egos there, pretty opinionated. As with any group of people there's the whole range of personalities. You learned that in college and in high school to perceive their stages of life. It's not like it was really thought about it a lot of the time but thinking about it you understand that the dynamic of personal relationships, the dynamic the work environment, people's opinions, the ability to work together, the importance of it goes up. 

You could say at school, “Well, that's fine it's my clique or my group and I'm in this. I'll play band.” (I’m not the sports guy, I was a band geek). You have your peer groups and as you get to working, not only do you have your peer groups but - as people would say now - “Your life depended on it!” Your success is how well you work through those mazes and those challenges. 

At the time, you didn't think about it that much but obviously when you talk about how was it was: It was casual and less stressful but also inherently behind that - and maybe not me recognizing it at that age - there was still at the end of day, “You're gonna have to make money and produce product and sell or the business isn't gonna be around and your job is not gonna be around.”

There's certainly a level of intensity and focus that built out of that experience and searching out of that experience. There was a natural growing up. There was a very energetic, a very exciting, interesting, and casual environment but also kind of a level of importance and - I’m not sure whether I’d call it stress or not - but probably focus. Trying to do good work, trying to be innovative. There wasn’t a lot of references for what to do because it was still very new, still there was a lot of focus on making something interesting and compelling that people wanted to play. The fun factor was obviously a big part of that. How do you make things fun? How do you learn and master?

\textcolor{interviewer}{Interviewer:} Who were some of the people in that original pinball division? I know it kind of faded out from there.

\textcolor{interviewee}{Rich Moore:} I’m trying to think of how it was at that point… The pinball group was also going through an evolution. Williams was the big competitor, the center of interactive gaming was really in Chicago with all the pinball companies and redemption equipment (little claws that existed, carnival games). So that was really the center of the universe, right? Trade shows were based in Chicago. I think Atari, almost like simultaneously when I started - although I think it predated a year or two - Atari decided to start the pinball group as well. 

[Side Note: Atari ran a pinball division from 1977 to 1979 which released games like The Atarians, Superman, and Hercules. The major issue that shuttered the division was an inability to attain pinball parts in California, therefore they were unable to compete with the Chicago-area pinball companies with existing relationships in their area.]

There were sort of opportunities to work in that but it was kind of a temporary or almost like a borrow situation, where I was part of engineering but on assignment work on a project that would be a pinball game. There was a lot of that kind transfer or a project assignment in pinball, and then once you finish that maybe you do another pinball game - I think I did - but it would not be that uncommon “Okay now time to come back home and let's do a video game.” It was fairly dynamic.

I think pinball was, particularly for new hires, a good way to kind of break in on coding out of school.

\textcolor{interviewer}{Interviewer:} Before you came into Atari were you a fan of coin games?

\textcolor{interviewee}{Rich Moore:} Yes, very much. My boss at the time at Hughes, who I went out with for beers and pizza a couple times, we'd go to place that would have a video game. Indy 4 or Indy 8 had just come out, one of the larger format games that allowed multiple people to play a race against each other. He really liked that. About that time Breakout had just come out, it may have come out my last year doing college. I’d play with my buds in college and who would buy the next pitcher of beer was who won the game of Breakout! It was a part of college and work around video games. To a certain extent pinball as well but certainly video games was kind of a captivating thing at that time.

\textcolor{interviewer}{Interviewer:} Do you know where Atari got their pinball technology from?

\textcolor{interviewee}{Rich Moore:} Some of it they kinda rebuilt from scratch. I think they probably hired some people out of the Chicago area, some people that would have formerly worked at Gottlieb or Stern or Williams or any of the other pinball companies. It turned out one of the guys that worked in that group was actually someone I went to high school with. A lot of the mechanical engineers themselves I think were hired out of San Jose State or Santa Clara, kind of local Bay Area as well.

\textcolor{interviewer}{Interviewer:} Steve Ritchie was there I know.

\textcolor{interviewee}{Rich Moore:} Steve Ritchie was there, he end up moving back to Chicago and basically was one of their founders for the video game section at Williams.

\textcolor{interviewer}{Interviewer:} Which pinball machines did you work on?

\textcolor{interviewee}{Rich Moore:} The first one I worked on was Superman. That actually got released. That was one of the first microprocessor based pinball games which Atari did. Then I worked on a really really large format game called Hercules which actually used a cue ball rather than the metal ball. And I think I started on another one.

My boss Steve Calfee was creating software libraries that could be used across different product categories. He built the base development environment as well as the base core libraries. In fact he designed a psuedo-language that could be used for pinball games. The coding I used - which is another reason here why pinball would be a natural good first first project - was really in a pseudo-language rather than having to code all the routines and the actions. 

It was relatively straightforward and in a higher level instruction to be able to create the rules. To say: You hit these three targets in order to get bonus 100 points, you hit these sequence of bumpers then this light goes on. All that kind of “business logic”, if you will, he had put together with a pseudo language that helped out.

\textcolor{interviewer}{Interviewer:} With those cue ball things, I’d certainly like to hear some of the problems with that but I'm also curious: Do you know who came up with that?

\textcolor{interviewee}{Rich Moore:} The design? I’d assume it’d be whatever the management team there, reviewing concepts. I was new to that. I had just been hired, so I wasn’t actively engaged in the game design process. We did usually one major brainstorming session each year, sometimes a couple sessions each year. In the pinball division, I wasn’t engaged with that at all. The presumption would be a very simple parallel to what existed at the video side which was some form of game brainstorming sessions: a process that would come up with ideas. We were owned by Warner Brothers so, you know, a lot of the games both in pinball were Warner properties. Superman - and in a similar way moving forward - a lot of video games came from licenses.

\textcolor{interviewer}{Interviewer:} What was kind of your role with those early games? Did you have a specific role or was it just “Here's a problem, have at it”?

\textcolor{interviewee}{Rich Moore:} At that time the video game team was really the hardware engineer and the programmer. Maybe you’d get a chance to have some type of creative person, some type of art component they help out. Pretty much the games were myself doing essentially everything from a development standpoint, outside of the hardware.

\textcolor{interviewer}{Interviewer:} Okay. So it was only those two pinball tables you did?

\textcolor{interviewee}{Rich Moore:} Yeah, that got released while I was working on it. I started on another one but did not finish it.

\textcolor{interviewer}{Interviewer:} Tell me a bit about your boss Steve Calfee. What was his sort of style of management and how did he bring projects to you?

\textcolor{interviewee}{Rich Moore:} He was an easy-going fellow as well. He went to Berkeley, so that’s common (quite a few people went to Berkeley so there’s a shared history there). Pretty easy to work with. I can’t think of him ever getting angry and yelling. Pretty matter of fact. If you came in and did your work, you’d generally be okay but you wouldn’t be successful. He was pretty good at challenging people to the level where they can handle it. You never felt like you were asked to do the impossible. He found a way to survive and structure things.

\textcolor{interviewer}{Interviewer:} That goes to the thought of, “We could do anything so nothing's impossible!”

\textcolor{interviewee}{Rich Moore:} [Laughs] Yeah, there was a bit of that. There were challenges for you but nothing that was insurmountable. Obviously what we were doing back then was physics approximations, or coming up those ways to make it seem gravity was working like in Lunar Lander. “Good enough” was probably not the phraseology used around how you meet the objectives with the tools, the processors, or the calculations available to you. 

At the end of day we end up finding and discovering – David Shepperd probably did a fair number of these as well - finding good approximations that could solve mathematical or physics problems. A lot of them have that nature to it. I think that was good and because there was good team dynamics and team morale. You felt if you were stuck, we generally had lunch together (most of the buildings had a cafeteria of some sort), and there’s the classic water cooler conversation: You're running against some problem, people were there to help you out and brainstorm. Someone could say, “Hey, I’ve got a neat bit of code or whatever that solves this thing.” They could say “Neat! I could use that.” A lot of sharing of comments and solving problems as a team. There wasn't a competitive aspect to say “Hey here’s a secret logarithm, he can’t see it! He can’t have that.” It was the opposite of that.

\textcolor{interviewer}{Interviewer:} How about a Howie Delman?

\textcolor{interviewee}{Rich Moore:} Howie? Well Howie was my project leader - as we called it back then - on Lunar Lander. I think it was John Ray on Red Baron. We ended up doing both the upright game and then we did the cocktail version (and Howie was much more generous). Howie was a pretty fun guy to work with. Funny character. Good, approachable fellow. Loved life - still loves life. Still has fun. 

He could get a little bit more emotionally engaged, get angry and frustrated than Calfee did. It was never overly long. Very, very few people there burst the lid. Certainly he was more emotional than Calfee and as a manager supervisor. Good guy to work with. Focused on solving the problem, not getting off on silly things, strange things. Very much focused on getting the job done in game development, making progress.

\textcolor{interviewer}{Interviewer:} I know a small contributor to Lunar Lander was Ed Logg. How was it working with him? He stayed there for a very long time.

\textcolor{interviewee}{Rich Moore:} Ed Logg was there quite a long time. He’s a very sharp guy. He actually optimized some of the stuff I'd done. He had access to all of my source. He took all the character design I actually came up with first. Lunar Lander was the first vector game [at Atari], so all the character sets I developed but he did it and made it better. He optimized some of it, found a better subroutine, and kind of defined the framework better than I had. Very very focused, very organized, very very sharp guy. 

That’s just one of the things where sharing and take advantage of stuff. In this case something I had done but he would evolve it, take it to the next level. At the tail end of when I was working on Lunar Lander, he was starting up on Asteroids. There was some overlap of development there as well. So there's no doubt probably stuff that I borrowed from him. No doubt that happened. It’s not as fresh in my mind.

There was not only the opportunity but really the practicality of people sharing stuff. Talking over lunch or having beers, or whatever it was. “Anything new you came up with this week?” or even code reviews. “I looked over some stuff I think there's something here. Maybe you can make this a little bit faster or better.”

\textcolor{interviewer}{Interviewer:} Now did you interact much with Lyle Rains?

\textcolor{interviewee}{Rich Moore:} Yes. I reported to him quite a few for quite a few years. I actually was promoted to VP of Engineering and Lyle reported to me for a little bit. Lyle has a pretty dry personality. Probably the most serious of all of them. You think about Howie and Calfee, certainly more serious than I am. More business-like, more of a manager personality from day one. It took me a while to work it up, I learned probably a lot more than I could have, certainly at the beginning. I had a lot of fun working with Lyle. A lot of challenges and frustrations but at the end of the day I still enjoyed working with him.

\textcolor{interviewer}{Interviewer:} I don't know if Lyle was from Berkeley but I know that Steve Bristow was.

\textcolor{interviewee}{Rich Moore:} I think Lyle went there too. Calfee, Bristow, Howard Owen...

\textcolor{interviewer}{Interviewer:} There's a lot of Howard's too!

\textcolor{interviewee}{Rich Moore:} Yeah! A lot of Howies, a lot of Howards. Steve was the big one. Seemed like half the people working there were Steve.

\textcolor{interviewer}{Interviewer:} So you started in pinball and you were doing programming there, but what you were doing was kind of synced with the hardware in a lot of ways. Did that, down the line, lead you more in interest towards hardware? That’s what I’ve seen from some of the VAX Mails later, you were more of a hardware guy, right?

[Side Note: The internal Atari email list which was contained on a DEC Vax mainframe was preserved by Atari’s Jed Margolin and made available online. Many of the questions addressed to Rich Moore in the late 80s from this email list were hardware-based.]

\textcolor{interviewee}{Rich Moore:} No, I was always a coder. Certainly a software guy, definitely not a hardware guy. I took hardware classes and I was familiar with and could kind of do it well, but I would not be the guy to work out the schematic and design the hardware. Probably not my skill or wheelhouse. I could read schematics and I could deduce, on some level, what was going on. But I was definitely a coder software guy. 

I worked on several art tools, like what we called the picture processing system (PPS) and then a lot of the utilities like the art generation or really large image file for version utilities. I did a bunch of that stuff as well. Coding the tools stuff like that was my area.

\textcolor{interviewer}{Interviewer:} To go into the vector games, had you seen any of the other vector arcade games out there before you started working on Lunar Lander?

\textcolor{interviewee}{Rich Moore:} Oh, yeah. Can’t think of the name of it now…

\textcolor{interviewer}{Interviewer:} Space Wars?

\textcolor{interviewee}{Rich Moore:} Yeah, yep. Space Wars. That one was big in college at the time, that was one of the first ones, right? I played that quite a bit in the arcade before. There’s probably one or two other ones that were out. We were all pretty much familiar with that. One of the things we do quite a bit at Atari is we bring in the competitor's products and when they got released, the new games. We would rotate it through our labs. We'd get those fairly early on, on release. It's almost like - we wouldn't quite get serial number one - but the distributor on the West coast, when they got their first shipment, one of them off the first truck would more or less come to our offices.

[Side Note: Moore notes here that he played Space Wars in an arcade during college, but the Cinematronics Space Wars game – the first arcade vector game – was not released until 1978, after he joined Atari. It is likely that Moore is thinking of Computer Space, a game that influenced a number of Atari alumni at the same time. It was not a vector game though.]

\textcolor{interviewer}{Interviewer:} Right. The vector system was developed up in Grass Valley. Do you know anything about who might've been the person at Grass Valley that actually did the hardware system?

\textcolor{interviewee}{Rich Moore:} I probably do remember their faces; their names are not fresh in mind. They may come to me.

\textcolor{interviewer}{Interviewer:} Because it was kind of detached, right?

\textcolor{interviewee}{Rich Moore:} Yeah, they were. Every once in a while they would come down to the Bay area and they kind of walked through and you would say, “Who are those guys?” “That’s Grass Valley engineering group.” You get a chance to meet them once in a while. There wasn't a lot of camaraderie. It wasn't so much that they were necessarily standoffish, but it just sort of like visitors from another division or group. You can say hello to and talk to them, that was fine, but they were sort of there to accomplish some business here, a couple of days, three days, and then they’d be back up there. Some interaction but not a lot of engagement, at least at my level.

\textcolor{interviewer}{Interviewer:} Did they have a microprocessor for the display and then you would have to have one for the game too?

\textcolor{interviewee}{Rich Moore:} No. There wasn't one specifically for the hardware. It was just the game itself. The game program would run on a processor. It was digital electronics with obviously a fair amount of analog because they were driving the beam on the CRT. There's components that obviously were analog, differential amplifiers and stuff like that, but no there's no processor. Those were only in the display itself.

\textcolor{interviewer}{Interviewer:} By the time you get the hardware in, has the Lunar Lander project started or do you get the hardware in and then Lunar Lander comes around?

\textcolor{interviewee}{Rich Moore:} That's a great question. The hardware was pretty far along at the time I got it. I probably had a prototype and the first prototypes we did with wire wraps. The circuit boards were wire wrapped. So the first one I probably had was no doubt was a wire wrap version, but fairly quickly it went to PC (printed circuit), that was a fairly quick process. That would be some matter of numbers of weeks or a couple months, three months. It was probably not too far along a project when I had a real PC. Probably not quite the production version, but probably one or two revs away from of it. 

I started on the wire wrap, but it would probably be the project on the wire wrap was kind of debug and working, kind of proven. The game project would be very, very close, so there wouldn't be a lot of gap there. It wouldn't be sitting on the shelf collecting dust. There would be be a game project on it pretty, pretty darn quickly. Again, I wasn't too engaged in the brainstorming process at that point, but no doubt there's been multiple concepts that have been talked about and kicked around.


The fact that Lunar Lander existed as a real game on the HP system, that was a pretty easy pick. You can look at that and say, “Hey, we can do that and this is hardware. We know the game is fun. The fundamentals are there. The enjoyment factor is there.” That's where Howie was taking the format, “How do you turn it into a game? So how do you monetize it? What's the game arc to it and what are you charged for?” Stuff like that. Yeah, brainstorming and I probably got involved a little bit on that, but at least in play, or at least in strong formulation. If not, it's pretty close to finalization between some of the basic coin-op aspects.

\textcolor{interviewer}{Interviewer:} But it was always a vector game.

\textcolor{interviewee}{Rich Moore:} Yes. Always a vector game. I did work on a regular, standard raster game right before that… Stretching my mind here. I worked on one other pinball game, it got transferred to Norm Addlebar. He had just been hired out of Davis - which is where I think he went - he was there for under a year or two. So I had started on third pinball, but that went to Norm. I was working on a raster hardware for a short period of time and then got onto the Lunar Lander, X/Y game.

[Side Note: Documents at the Strong Museum of Play in Rochester, New York hold a number of early Atari project reports dating back to the mid 70s. Several memos from around 1975 show that Atari had been working on a raster-based version of Lunar Lander for several years before the game was revived with Howard Delman and Rich Moore.]

\textcolor{interviewer}{Interviewer:} So you were actually working on the hardware?

\textcolor{interviewee}{Rich Moore:} To answer your question, when the prototype vector generator hardware was ready, I moved to that. When you can start coding it, programming it, move stuff around, I moved to that when it was at that stage.

It's almost like right out of hand. All the engineers say, “Yeah, check out all my logic and everything works correctly.” Calfee, literally that second. Next second, “Rich! Need you to start putting the games together.” [Laughs] But like right before that I was working on a raster hardware. 

I can't even remember the game design concepts. I was working on a raster hardware. I got the development system set up, could assemble programs, write programs, get them running, and be operational. Interrupts, handlers, all that kind of good stuff. Then moved to Lunar Lander virtually when the hardware the hardware engineers debugged it. We literally took the baby away from and said, “Okay, now it's time to program. Let's do something!”

\textcolor{interviewer}{Interviewer:} [Laughs] Okay. What was the development system at that time? Cause I know you guys went through a few over the years.

\textcolor{interviewee}{Rich Moore:} Yeah. It was floppy disks, and they were like the eight inch format floppy disk. You'd load that into an advanced system and then you could download that to RAM that was in the place of what would eventually ROMs. The development boards were all RAMs. Hit the reset button on the board, have it kick off, then try to run the program.

\textcolor{interviewer}{Interviewer:} Was it the PDP system?

\textcolor{interviewee}{Rich Moore:} Yeah. We had two ladies that would code in your program. You basically would write it out on pencil on paper and then read lines as she got a print out. They would just code all that in. They would actually type all that in onto the floppy and run the assembler. You'd have your source text files and code, then you'd have the assembler executable that could get loaded into the hardware.

\textcolor{interviewer}{Interviewer:} I think the number that I've heard in terms of cabinets for Lunar Lander was about 20,000. Does that sound somewhat accurate?

\textcolor{interviewee}{Rich Moore:} Oh gosh, many years ago. That sounds about right.

\textcolor{interviewer}{Interviewer:} Yeah. I know it wasn't as big as Asteroids, obviously.

\textcolor{interviewee}{Rich Moore:} No, Asteroids was obviously the killer. My claim to fame was, that was the first one we did, the first X/Y Vector game. Obviously towards the tail end of that production cycle, it consumed the production lines for many, many, many months.

\textcolor{interviewer}{Interviewer:} Then they had the had the upgrades with Asteroids Deluxe and all that. Then after that, they're working on Battlezone. Did Red Baron come about while Battlezone was being worked on or was that completely afterwards?

[Side Note: Battlezone and Red Baron were brainstormed at about the same time. Red Baron took longer to develop and came out in 1981, whereas Battlezone released in 1980.]

\textcolor{interviewee}{Rich Moore:} Once we got into Battlezone, I did a small game called Red Baron. At that point we had added a kind a bit-slice math component - actually a sub board - that allowed running batch matrix operations. That was a new hardware specification and that was used for Battlezone, Red Baron, several game developments that were never released, and then obviously the color version of that later went to Tempest (but it's a really different hardware set/platform then what was used for Lunar Lander and Asteroids).

\textcolor{interviewer}{Interviewer:} Right. Who was the person that sort of came up with the 3D technology? If you know. Or was that the Grass Valley guys?

\textcolor{interviewee}{Rich Moore:} There were quite a few people that kind of participated in that. I think probably Michael Albaugh and/or David Shepherd probably did some of the bit-slice coding. I'm trying to remember now. The hardware engineers, I’d probably give the wrong credit there. I would guess Howie Delman had some engagement with it, but I just remember it was a team effort. I'm sure there was a principal engineer that owned that design, but it could very well use internal consultants or team to break apart some aspects of that hardware. Different sections to different people. 

Rick Moncrief was pretty heavily engaged in some of that tech. In my memory he served as the interface to Grass Valley and some of the other R\&D stuff. Of course, that’s where the Vector components came from. He was Applied Research so he would pick up some of that stuff from a transition standpoint, I guess, and then really in all cases there would be an assigned hardware engineer that owned that particular project. Whether it be a Battlezone or Red Baron or what have you.

The Red Baron project manager was John Ray. He was responsible as a project manager of the project. I did all the coding. Then Ed Rotberg got Battlezone. Pretty sure he did all the tank models, although I'm not sure if he may have used some art resource for some of that stuff.

[Side Note: Howie Rubin actually drew up the tank models.]

\textcolor{interviewer}{Interviewer:} I don’t know when Rick Moncrief was part of Atari. Was he there for pretty much the time that you were there?

\textcolor{interviewee}{Rich Moore:} Yeah, he was. He was there before I was, I think in the coin op group. He was responsible for a Hard Drivin’ and that was really successful. He was there before I joined and had Applied Research, essentially an R\&D group. They certainly had their own projects with the Hard Drivin’ being the principle one as a development group.

\textcolor{interviewer}{Interviewer:} So with these kind of early 3d games, how would you kind of map out the surroundings? Would you be doing, like you do in ROM where you do a top down thing where you mark out “Here's where the planes spawn from. The mountains over here.” I forget what it is in Red Baron, but I know in a Battlezone there's these big triangles that form obstacles. I don't know if there's any permanent objects aside from the mountains in the distance in Red Baron.

\textcolor{interviewee}{Rich Moore:} Yeah. Well in both of them there was an “atmosphere sphere”, like the horizon.

\textcolor{interviewer}{Interviewer:} We call it a skybox now.

\textcolor{interviewee}{Rich Moore:} Yeah, exactly. I vaguely remember we had a bunch of graph paper, I used that to map out the three coordinates for all the different planes (it may have been a plane). For Red Baron, you were more or less on a rail so almost like you're on a race course. There were objects that were there that may have added some somewhat to the atmosphere, but it'd be almost be like programmatic: a formula to say every so often, based on speed or something like that. 

You were on a race course, really on a rail. You could go left and right, but the game in terms of the coding was going to force opponents to come at you. Some relative X-Y translation on the screen, and then off in the distance the velocity or acceleration would be determined by the game speed. I think we sped it up over each level.

In Ed's case… I'm trying to remember here. There were landmarks on his grid. At some point you would have had to have a big grid graph and say, “What's my universe? It's this.” I think his world rotated in on itself. It was the extra connected curve right here on this strip that wrapped around. Pretty sure it kind of repeated that way. He had some of the pyramids and some of the other landmarks, so he had some reference points every X number of miles (or whatever you call the units there).

In terms of triggering opponents, he would introduce things in different parts. That may have been scripted at some point, I'm not sure how variable it was. You did want to follow that cause it could have something else that didn't show up on your left or right, you’d want to pivot around and face them while you’re being shot at. Like you said, it was a skybox essentially (or a mountain-range box). There was always scenery off in infinity and then there was those reference points, landmarks you would rotate around. They were specific locations, but it rotated after you got to the end of the edge of the world and you basically got sent to the beginning of the world on the other side.

\textcolor{interviewer}{Interviewer:} I'm sure you've heard the stories about players trying to reach the volcano in Battlezone. Was there anything like that for Red Baron? Any interesting behaviors that you observed?

\textcolor{interviewee}{Rich Moore:} Not really. At that point we were almost at the beginning of starting to think about Easter Eggs and but not any real planned intent there. We were more worried about if you did something that you may not take it or debug it completely and may come in when you don't want to come in. [Laughs] I think we were more paranoid about putting in little secrets that could hit you in the face because it wasn't really designed as well as it could have.

There’s the story with the original version of Asteroids where you could get all the free lives and the game would slow down because the refresh rate would slow down, so you had more time to react. It was probably at that point where we had the concept of a surprise or an Easter Egg was starting to show up. I don't remember really having anything in Red Baron myself… I should say, I don't remember having anything intentional in the design. [Laughs]

\textcolor{interviewer}{Interviewer:} The staff listed for Red Baron are you, John Ray, Jed Margolin, and Joe Coddington. Does that sound about right?

\textcolor{interviewee}{Rich Moore:} Yeah. Joe Coddington would have been the tech. Actually, that kind of answered the question you had earlier. “Who was responsible for the math engine?”, that would have been Jed. He did do the bit slice. The Applied Research team, Rick Moncrief’s team, Jed would have been responsible for that bit slice component that plugged into the board.

\textcolor{interviewer}{Interviewer:} He's a smart guy.

\textcolor{interviewee}{Rich Moore:} Jed is a very smart guy. Real interesting fellow. I have a lot less history with him, but a great, very bright guy. Very good guy, talented engineer.

\textcolor{interviewer}{Interviewer:} And I know that later he kinda took the vector hardware and he made his own kind of flight game that they never actually put into production. He has a lot of that stuff on his website.

[Side Note: On Jed Margolin’s website you can see his final ambitions with game called TomCat, which would later morph into Atari’s 3D hardware with Hard Drivin’. https://www.jmargolin.com/tomcat/tomcat.htm ]

\textcolor{interviewee}{Rich Moore:} Yeah, there's always all sorts of little side projects. Some little concepts or stuff that maybe people would work on, almost like the test bed or the verification for the hardware before it got released to the team and was really a game. Software would run it on and an application would load on it. There may be some little tricks a little bit or demo that the team or the designer put together.

\textcolor{interviewer}{Interviewer:} It was kind of a collaboration on Lunar Lander and then on Red Baron. You were pretty much the sole guy under direction from John Ray. Did it feel like a leap, that you were getting up in the Atari wrung?

\textcolor{interviewee}{Rich Moore:} Yeah, I think so. I mean, Lunar Lander wasn't my first game, but first video game. It was a step for a new guy, a rookie. So once you get your first one out then your second one, you definitely felt more confident. I guess work you work your way up in the organization.

\textcolor{interviewer}{Interviewer:} I am curious a bit about how the design kind of came together. How much did you contribute to the development of the market research in this early time? How did the designers talk with each other and how to interface with the leaders in the coin op group? Gene Lipkin and Frank Ballouz.

\textcolor{interviewee}{Rich Moore:} If I remember correctly, the game concept was probably John Ray. He probably came up with it (it may have been someone else). They would have pitched it to the annual game brainstorming session. I think all the execs would have been aware of it. In terms of projects that got approval or work their way up, concepts and stuff like that, the would have been aware of it. “Here's the stuff that's working its way up. Our queue projects and things like that. 
I wasn't engaged with those, I wouldn't really know what was happening. There were standard reports we were doing on projects like project status (that's where the project manager came in). Then, what are the new, pending concepts or pending game ideas that are next at bat, being considered for the next project. They would certainly weigh in and have their opinion, recommendations and stuff like that.

There were regular sessions (I can’t remember how often we had it) I’ll say it was monthly. We'd more or less have a demo of the current games and the sales staff would come in. We also had product dean - it was like the ongoing joke - if you ever saw a sales person walking through, the labs quickly hit the kill switch to just turn off the display. [Laughs]

\textcolor{interviewer}{Interviewer:} That was the Lipkin switch, right?

\textcolor{interviewee}{Rich Moore:} Yeah. To me it was almost like a bit comical, if it's like more of a joke than a reality to me. I think probably other people took it more seriously! [Laughs] We did have formal times where intentionally they’d show up, come through the lab, and see what's the current state of the games.

\textcolor{interviewer}{Interviewer:} Yeah. There that, and then there’s also the Stubben Test, right?

\textcolor{interviewee}{Rich Moore:} Yeah, yeah. I remember Ed Rotberg with Red Baron would come in, turn away from the display and just shake the joystick back and forth and fire. Particularly at the early part of the game, I more or less forced the first few planes to come right at the player. So like you could just do this, hit the fire button, you're going to blow up a couple enemy planes. He liked pointing that out every once in a while. It's kinda funny.

\textcolor{interviewer}{Interviewer:} One of the issues we talked about a lot with the early Atari creative contribution is the idea of accreditation. How present was that issue there? Was that something that was brought up a lot?

\textcolor{interviewee}{Rich Moore:} It's interesting how that developed. You can almost relate it to any creative enterprise, right? It's sort of like the same thing that happened decades before in movies, then TV, then there’s this whole thing with unions - SAG (Screen Actor’s Guild) and everything else were formed. The other thing I remember was DVDs and pay for play and all that stuff coming in. Strikes from writers in Hollywood happened. So creative accreditation, that concept or issue you can say applies to every creative enterprise.

We probably talked about it a bit internally. I think that the position and the understanding if you're management is, “I don't want to put people's names on a screen, because then other game companies are going to come in and know who to call.” The natural defense and the natural reasoning from the company standpoint was, “By doing that, we’re opening ourselves. We're basically just giving our staff contact information, if they’re a headhunter they can go after guys.” 

I think what happened, when Activision split off - I’m not sure if my timing’s quite right there - I recall that accreditation, whether it was in coin-op or consumer or both (probably coin-op at that point) started showing up. We were certainly not the lead that did it. It started showing up with some other company, whether it was Williams or Bally I can't remember now. Someone else did it first and then it sort of became the situation that, “Well, if they're doing it, you know, why can't we do it?” It's sort of like the cat's out of the bag, right? You can't really have an excuse for doing that anymore and we think we deserve it. 

That's a side of the business, as success grew and the company was always making money on conversions to consumer, so natural human ego [arises] and they want to get a little notice for being part of that.

\textcolor{interviewer}{Interviewer:} You know, the sense that I've gotten is that the reason it didn't really come to a boil was the fact that everybody was treated well anyway.

\textcolor{interviewee}{Rich Moore:} Oh yeah. It was a fun place to work at. I think there was a generally good feeling. It came from its roots of the sandal and shorts and t-shirt environment when it got established. It definitely wasn't aerospace and Lockheed or IBM with a white shirt and a tie. It was definitely a more casual, fun place to work at with an environment that wasn't high stress, although there were points where you had that. Stuff that happened and deadlines and stuff like that. I think some of that's true. 

I think the other thing that's actually very, very true is that everything was limited in the early days, particularly on memory size. Every byte mattered. All that directly leads to memory costs, memory was still scaling up to the point where you didn't have to be so concerned. We were still programming in assembly language, so it'd be as efficient as close to the nose as you can. So I think to put names in there and have them scroll like, I need another ten/twenty thousand - whatever it may be - I need more memory. That means I’m putting memory against this stuff; that means I’m compromising the depth of the game. I just call a lot of it being kind of technically limited just on that efficiency. 

Once we got into 3D games we were on the bit-slice, we had moved to the 68000, we were programming in C with a compiler so we had more memory available. Then you can not make the argument and say, “Oh, we don't have enough memory.” or, “If we add these we can add another two or four bucks to the cost,” that sort of went away too.

My personal opinion, I'd probably put it more to the technological aspects where other people may pick it up more as emotional and psychological. The soft skill human aspects of me look at it more than that. I just always kind of looked at it logically.

\textcolor{interviewer}{Interviewer:} Yeah. Like if Al Alcorn could have put his name on Pong, I think he would have, but he just didn't have the memory to do it.

\textcolor{interviewee}{Rich Moore:} Yeah, I mean certainly Al is a pretty strong personality, pretty big guy in history. He may have, I can't speak for other people.

I never felt I was on the stronger end of the ego side of the equation. I mean, everyone has egos. I never felt like I took it and part of it, to be honest. The killers there were people like Ed Logg and even Rotberg, later Mark Cerny and other people that had the top 10 games. I was probably a fair journeyman and in the middle of the pack, sometimes above average sometimes below average. Not up at that stratosphere, claiming a hundred thousand run of a game.

\textcolor{interviewer}{Interviewer:} One of the reasons I asked this is because I try to make sure that everybody is duly credited on things and Atari’ s pretty good about this. The games that I have you listed as direct contributions on are Lunar Lander, Red Baron, Marble Madness, Road Runner, and Race Drivin’.

\textcolor{interviewee}{Rich Moore:} Yeah. We produced a particular hardware later which I helped create. Coin-op operators were interested in a programmable platform and that’s where System 1 came from. Marble Madness was the first one on that platform. I was a manager/director at that point so I cross-matrixed managing teams. I did some technical stuff to support Mark Cerny. He needed some tools and things done. I certainly contributed to the game, but my work contribution was more around the tools and the production pipeline, not the game code. I don't think I did anything to the game code per se.

I was also responsible for the PPT, a Picture Processing Tool. I supported that tool and the infrastructure in the studio, and then I'll support a number of conversion tools that would take a dozen images and the artwork and convert it into the various forms and go into ROM. To some extent I could say any game that was produced over that period of time, they support that tool that was raster. I had some… Touch is the wrong term, but I was in the back office, back tech.

Hard Drivn’, I was VP of engineering at that time. That was Rick Moncrief’s, that's clearly Rick and his team. Those guys busted ass, put a lot of blood, sweat and tears on that. From an overall manager standpoint and at least having responsibility that stuff happened I could do that but I wasn't coding so I wouldn't have contributed code in there. The overall teams are the executive producers on stuff. As VP of engineering, that was under my watch then. I can say it's during my period there.

\textcolor{interviewer}{Interviewer:} Right. After Red Baron you started to get more into those leadership positions. Can you tell me how that happened and how that changed your interfacing with both the engineers who you were peers of, and then the management who you ascended to?

\textcolor{interviewee}{Rich Moore:} Yeah, I mean, it's a lot of stuff. Looking back at I was still relatively young, I'm not that experienced as a guy. There's a lot of things I look back that I did pretty poorly, I’m pretty critical of myself. I could have been a lot better. It changed the relationship, in a lot of ways. In some ways more my perception than necessarily realities, I probably translated this stuff more hardcore and more impactful than it really was. That was more in me, I think, then necessarily others. 

You know, now you have responsibility that you have to influence and try to control. Sometimes you control it well, sometimes you don't. Some of my techniques deployed were good, a lot of them were - looking back at it - bad and weak, things I should have done better. I still spent a lot of time in the trenches walking through labs, talking to people, having lunch with people, and participating. I'd run brainstorming sessions and be responsible for that. I still had a lot of engagement, but a portion of my time went to management things, budgets and teams. 

Unfortunately, we sort of had a reputation of having a layoff every three or four years, “rightsizing”, so I had to manage a couple of those which were obviously very unenjoyable activities.

\textcolor{interviewer}{Interviewer:} I know that you did project management with some of the imports Atari brought in from Japan. There was the game Arabian, and I think maybe the Kangaroo game too.

\textcolor{interviewee}{Rich Moore:} Yeah. A number of those got... I can't remember all those now. There's that one group out of Boston, Massachusetts who we did a couple of games with. There was that racing game... I did have my hand on that when I was manager or director there. Part of that is again, sort of time availability. We had skillsets, so it's the license thing. I can kind of manage that project in addition to the other projects which had their own project managers on it.

\textcolor{interviewer}{Interviewer:} What were you doing with these? Cause you know, in a logical sense, there's not a whole lot to do once you get the board in. What are you doing to facilitate manufacture of these foreign games?

\textcolor{interviewee}{Rich Moore:} Fundamentally you’re sort of resolving the licensing aspects of the business. You comply with the business aspects of completing the license, reviewing that agreement with legal, but really we're getting that into the pipeline. You want to keep your running field tests, collection tests, find out what kind of performance, making changes or recommendations to the game design. 

There almost always would be some type of feedback back to the licensee, sourcing “We need to have this kind of stuff.” Then you'd manage the whole cabinet design process of the artwork. “Which cabinet are you gonna use? What else?” All that sourcing stuff, making sure that you have the right power supplies, part selection, the right components.

\textcolor{interviewer}{Interviewer:} In terms of, the interface back with the, with the licensee was, is that like translation, stuff like that they would change?

\textcolor{interviewee}{Rich Moore:} In some cases. Yeah. I mean, obviously I have to have the English version of it, if it was from Japan or something.

\textcolor{interviewer}{Interviewer:} Who was the management head of coin-op by the time you were there? Was that Ken Harkness or John Farrand?

\textcolor{interviewee}{Rich Moore:} Well, they both were. John Farrand was responsible when we were releasing System 1, so I actually had a meeting with them back in Japan and pitched it to Namco. They were both there. Then if I remember right, I think when Warner purchased us, John went to another business inside the camera team company inside Warner. I was familiar with both of them. At the time that I was VP I was actually reporting to Hideyuki Nakajima, president at that point.

\textcolor{interviewer}{Interviewer:} That’s a whole different era of Atari so we’ll stop there. Thanks so much for your time and memories!
165 days does not seem like much of a time to make an impact on the industry, but Gregory Cox happened to be hired at the Atari-affiliated (soon owned) Cyan Engineering at the perfect time to leave a mark. Located in Grass Valley, California, Cyan was an advanced research and development group which worked on Atari’s latest and greatest for the first decade of the company’s existence. Their projects would include the likes of the Atari Video Computer System, the Atari 8-Bit Computer line, and arcade games like Starship 1.

In early 1974 Cyan had just wrapped up creating their first arcade hit for Atari, Gran Trak 10. It was among the first games in the arcade to use a ROM chip, a specific computerized component among an otherwise muddy mix of digital and analog hardware made to create machines back then. At the end of the prior year, Cyan and Atari had been approached by Intel to explore the potential of using microprocessors as a part of video games. Intrigued, they bought themselves an early Intellect 4 development kit for the Intel 4004 range, with the intent of building some prototype projects.

Out of Ampex they hired the young programmer Gregory Cox to tackle this new challenge. They would set him loose with the development kit and a Bally El Toro pinball machine with the task of recreating the functions of the moving relays with a microprocessor and program code. After this he would also work on at least one video game prototype before leaving the company in the middle of Atari’s major financial difficulties in mid 1974. His labors potentially produced some of the first microprocessor based games, so let’s have Mr. Cox take the story from here.

\noindent\makebox[\linewidth]{\rule{\paperwidth}{0.4pt}}

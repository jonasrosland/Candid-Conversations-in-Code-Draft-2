Within history, there is always a deep desire to identify a ‘first’ as to put all other developments in context. All of the stories in this book represent the very epoch of professional programming for video games, but the first game to use a microprocessor has been long identified as a defining moment for video games. The long-claimed holder of that title in the arcade has been Gun Fight, released by Midway in November of 1975, and may still hold it’s place depending on how one defines the question.

The game was developed by the firm Dave Nutting Associates (DNA) in Milwaukee, Wisconsin, run by the eponymous Dave Nutting. A rather visionary creator who entered the coin-op industry well before video games, Nutting and Jeff Fredricksen established DNA in 1974 to take advantage of the new microprocessor technology emerging onto the scene. They created a feasibility pinball prototype using Fredricksen’s expertise, but to move forward they would necessarily need dedicated programmers to develop this concept further.

Fredricksen had a connection with the nearby University of Milwaukee and the robotics instructor Robert Northouse. Northouse would provide the opportunity to work at DNA to several of his students, the first of which being Tom McHugh. McHugh was a precotious young student who self-admittedly didn’t care much for games, but as a problem-solver his dedication was second to none. For the first time he tells his story of how he came to create the first microprocessor games, how he became one of the first work-from-home developers, and why the Arcade Video Game Crash scuttled his potentially vibrant career in the sector.

\noindent\makebox[\linewidth]{\rule{\paperwidth}{0.4pt}}

\textcolor{interviewer}{Interviewer:} Have you always been in Milwaukee?

\textcolor{interviewee}{Tom McHugh:} Yes.

\textcolor{interviewer}{Interviewer:} Were you born there?

\textcolor{interviewee}{Tom McHugh:} Yeah, I was born there. I spent some time in the military, but other than that I’ve been in Milwaukee – Wisconsin, per say.

\textcolor{interviewer}{Interviewer:} What did you serve in?

\textcolor{interviewee}{Tom McHugh:} The Army.

\textcolor{interviewer}{Interviewer:} Okay, just the straight forces. Just as a regular soldier?

\textcolor{interviewee}{Tom McHugh:} Yeah. I was – You know what the NSA is?

\textcolor{interviewer}{Interviewer:} Yeah.
\textcolor{interviewee}{Tom McHugh:} Each one of the military services has course plan military version of that which is typically for the purpose of gathering signal information. There’s usually antenna farms which listen to other people are doing and saying, I was part of that.

\textcolor{interviewer}{Interviewer:} Did any of that involve electronics work or was that purely just recon and things?

\textcolor{interviewee}{Tom McHugh:} No, I was really in the signal intelligence part which transferred messages between different places. A lot of the stuff went to the NSA.

\textcolor{interviewer}{Interviewer:} When did you first get interested in electronics then? Was it before that or was that kind of your first exposure?

\textcolor{interviewee}{Tom McHugh:} That really wasn’t the kind of exposure, to tell you the truth. After I got out of the military, I went back to school and discovered programming and really loved it. That was how I got together with Dave Nutting and Jeff Fredricksen.

\textcolor{interviewer}{Interviewer:} What were you looking to go to college as if you weren’t- Because I don’t imagine you were going there to do computer programming.

\textcolor{interviewee}{Tom McHugh:} Yeah, I was. I took a course on it and I thought, “Oh, I really like doing this.”

\textcolor{interviewer}{Interviewer:} Okay. I wasn’t fully clear on that. How did you first get exposed to computers then?

\textcolor{interviewee}{Tom McHugh:} Well I went to UW Milwaukee and that’s where I got exposed to it.

\textcolor{interviewer}{Interviewer:} Was that the course with Dr. Northouse?

\textcolor{interviewee}{Tom McHugh:} Yes.

\textcolor{interviewer}{Interviewer:} Can you tell me a bit about him?

\textcolor{interviewee}{Tom McHugh:} I can tell you anecdotes but I can’t tell you very much about the person.

\textcolor{interviewer}{Interviewer:} Yeah, right.

\textcolor{interviewee}{Tom McHugh:} I think he went to Purdue, that’s where he got his degree, his master. One of the anecdotes was that he went to the computer gathering of ‘whatever’ in New York. They have companies set up a little area, and one of them had a big sign up and said “Can you guess how much our profit was last year?” and the person who guessed closest is going to win a minicomputer. He was the one who got the closest, I guess he missed it by two dollars.

He came back to Milwaukee and said, “Hey I just won this minicomputer” and the people who were there said “Oh. Now what do we do?” So they came up with this concept, they called it the RAIL Lab, “Robotics Artificial Intelligence” which is like way beyond what it was really doing. He set up this lab which was basically these two large rooms next to each other that had stuff in them. One of the rooms, a lot of the MBA students were working in there and he had this big elaborate train layout. The idea was that it had switches all over the place and people would do their master’s thesis on switching stuff, switching theory for an example, on this massive train layout. Which is kinda weird, [Laughs] but that’s what they were doing.

Anyway, I came along, I decided to go into the master’s program and I was looking for something to do. There was no software people that would take me on because there was no software people there! So then there was Northouse and I said “Would you be my advisor?” and he said, “Sure, what do you do?” I said, “Software.” He said, “You know what? We don’t have anyone here that does software, we’ll take you on.” You can imagine what happened then. All the software came my way. “You do this? You get to do it. You do that? You get to do it.” That’s where Northouse came in.

Jeff probably told you he took a course from Northouse. Did he say something like that?

\textcolor{interviewer}{Interviewer:} I haven’t been able to talk with him fully. He works at Apple now. I talked with him briefly and he said you would be a guy to contact, basically.

\textcolor{interviewee}{Tom McHugh:} [Laughs] Thanks a lot, Jeff! No, I don’t mind, I don’t mind.

So actually Jeff and Dave Nutting had this place in Milwaukee and they were trying to build something and they needed some software people. They went to Northouse and Northouse came to me and another guy he had working there – he was a hardware guy but he knew a bit of coding. Basically he hired us out to Dave and Jeff to start working on their very first game.

A little ways after that the other guy decided he really wanted to go back to school, he was basically teaching there. He had his MBA and he was doing some lecturing, he thought that was more fun than this other thing that Dave and Jeff were doing. I thought it was great! That’s how we got together.

\textcolor{interviewer}{Interviewer:} Just one other question about the RAIL lab, what sort of minicomputers were you using there?

\textcolor{interviewee}{Tom McHugh:} It was called a MODCOMP Model 3 or something like that? It was one of those things that had a disk in it, 256K disk in it or something (and that was back in the 70s). It came in 3 consoles that were kind of next to each other. It was pretty good computer but they didn’t have much use for it. They did something, they were building a robot man, they ended up having like a tether for it. They didn’t get to a point of separating the robot from a computer.

\textcolor{interviewer}{Interviewer:} Did you meet Fenton before you got into the program? When did you meet him?

\textcolor{interviewee}{Tom McHugh:} Fenton was always just kind of there. I don’t know if we ever had classes together. I knew of him, I saw him around cuz you know we went the same circles and I’m not sure which circle he was in, but he wound up with Dave Nutting. I think he started off doing pinball machines using an F8 Fairchild processor. There were a lot of companies doing stuff like that back then before they started consolidating.

\textcolor{interviewer}{Interviewer:} Did you know that Fenton was also part of Northouse’s class or you just didn’t even know that?

\textcolor{interviewee}{Tom McHugh:} No, I didn’t know that.

\textcolor{interviewer}{Interviewer:} Heh. What he said is that you both came from the program and I thought that meant you came together, but I guess [Laughs] you just didn’t even know each other.

\textcolor{interviewee}{Tom McHugh:} It was another guy that I came with was a teacher there, he was an instruction. Then he decided half way or a quarter of the way of the first thing he worked on that he really didn’t want to do that and he basically said, “I’m not gonna do this anymore.” He dropped out and then it was just me. I don’t know if Fenton came on later, after that. I think he did, but I don’t remember.

[Side Note: McHugh only knew Fenton when she presented as male in the mid-to-late 70s.]

\textcolor{interviewer}{Interviewer:} When you went down to where they were stationed, what was your first impression of Jeff and Dave?

\textcolor{interviewee}{Tom McHugh:} I thought both of them were two really serious people. My impression was that they knew what they were doing and whatever they were going to do they could bring on. Some of the things about Dave was [he] was pretty wealthy I guess. He was an industrial engineer and he would make things, some small number of them, then they would stop selling, he would go bankrupt, start another company, and make something else.

\textcolor{interviewer}{Interviewer:} [Laughs]

\textcolor{interviewee}{Tom McHugh:} The same thing would happen over and over and over until he got together with Jeff, and I don’t know how he got together with Jeff. Then they started looking at doing video kinds of things because some of the other things that Dave did were not really video, per-say.

\textcolor{interviewer}{Interviewer:} Right. That’s funny the way you described that because that’s what I’ve seen. He had two companies before that and they both went bankrupt. [Laughs]

\textcolor{interviewee}{Tom McHugh:} It was just something for him to do because he had enough money but he needed something to do with his life. So he would do this and he would do that and the thing that he got into really wasn’t that big of a demand. The thing is when he really started working with Midway that was the same kind of thing almost, in the sense that there was a larger organization but they would build these consoles that they sell 2,000 or 3,000. That’s it, then you go onto the next thing. It was the same kind of thing except that they sold a lot more [Laughs] and he didn’t go bankrupt.

\textcolor{interviewer}{Interviewer:} [Laughs] Were there any other employees there at the time when you came in to meet Jeff and Dave?

\textcolor{interviewee}{Tom McHugh:} Do you know where they did this, the place itself? They started out in this place in Milwaukee and that’s what I’m asking you, are you familiar with that place?

\textcolor{interviewer}{Interviewer:} Well I know Dave Nutting Associates the company, I‘ve never seen the building or anything.

\textcolor{interviewee}{Tom McHugh:} Okay. The building that we all started in was a warehouse. It had this thing in the front with two or three offices and there was some people in these offices. There were three people in this organization and this organization was called Red Baron. They did something- Do you know anything about that?

\textcolor{interviewer}{Interviewer:} Yeah, that was an offshoot of Dave’s previous company, they were running arcades. [Note: The previous company was Milwaukee Coin Industries. A subsidiary, Red Baron, owned a warehouse for games which Dave Nutting owned personally even after he left the prior company.]

\textcolor{interviewee}{Tom McHugh:} Yes, okay. There were actually I think four people there, because what I’m leading to here is that there was a president of Red Baron then I think there was a secretary and a bookkeeper. Then there was a fourth person who was like a fixer of arcade machines, of pinball machines. That fourth person (and I forget what his name is) he was the other person that worked for Dave Nutting. He was like a technician. Dave would say, “I wanna build this thing.” Jeff would come up with the design of it and start building it. This other guy would do a lot of the soldering, for example. He was there, I think, on day one.

This building where we were in, like I said, the front part was these several offices and the middle part was like this big open area. Apparently what happened was Red Baron had a lot of pinball machines out in the city of Milwaukee. Then for some reason, Red Baron was told they had to get all the pinball machines out of places that they were in, so they took all the pinball machines and put them in this warehouse. That was the middle part of this warehouse. Then in the back there were a couple rooms and that’s where we were. We were in the back. The cool thing about the pinball machiens was they were all set up that you just go out there and you flip a button and you could play them. That’s one of the things that we did a lot of.

\textcolor{interviewer}{Interviewer:} Nice. With those people there, I know a couple names maybe they’ll ring a bell. DeWayne Knueston?

\textcolor{interviewee}{Tom McHugh:} That’s him. That’s the guy that I was trying to remember. What other names do you have?

\textcolor{interviewer}{Interviewer:} Dan Winter I think was the president.

\textcolor{interviewee}{Tom McHugh:} Yeah, could be. He never interacted with Dave Nutting Associates, really. It was really all about Red Baron.

\textcolor{interviewer}{Interviewer:} He kind of split off, didn’t really do that. Then one other name Joan Mason, was that the secretary?

\textcolor{interviewee}{Tom McHugh:} Yeah, she might of been. That doesn’t ring any bells. Like I said, they did their thing and we did our thing we really didn’t cross paths much.

\textcolor{interviewer}{Interviewer:} You just shared a building?

\textcolor{interviewee}{Tom McHugh:} Yeah.

\textcolor{interviewer}{Interviewer:} Did it seem kind of weird, the idea that they were setting up a contract game manufacturer? Had you ever heard of that, a different company would make a game for someone like Bally or Chicago Coin?

\textcolor{interviewee}{Tom McHugh:} Not at the time but a lot of those games came from Japan too it became that way. At that time, now.

\textcolor{interviewer}{Interviewer:} Were you ever a big game-player yourself whether it be coin-op games or regular games?

\textcolor{interviewee}{Tom McHugh:} Not really. I mean that’s not really why I was drawn to it. It was just an interesting thing to do.

\textcolor{interviewer}{Interviewer:} Though by that time […] had you seen the early video games that were starting to make inroads?

\textcolor{interviewee}{Tom McHugh:} Well depends what you mean by that. The only other one that was really around was Pong. The thing that Dave Nutting started was actually the very first video game that was run by a computer, a microcomputer as it was called at the time. There wasn’t anything that could compare to that, really.

\textcolor{interviewer}{Interviewer:} Right. Before they were doing it with the video games they also had the pinball machine that you mentioned. Did they show you that pinball machine?

\textcolor{interviewee}{Tom McHugh:} I don’t know if it was before, it may have been after because that’s when Fenton came in. That’s why I said Fenton came along later. They started doing this thing with the Intel processor to do video games and that’s how they got on the Fairchild processor to do pinball.

\textcolor{interviewer}{Interviewer:} They had a couple other microprocessor-based experiments. Did you see any of those or did you only ever see the video game? Or were you only ever told to do the video game? […] They were experimenting with microprocessors as controllers for several different machines. Did you ever see any of that or were you told ‘Video game is all you need to focus on’?

\textcolor{interviewee}{Tom McHugh:} Well I didn’t really care that much about the pinball machines. That’s the only other thing I was really aware of. Video games on the one hand and video games on the other, I don’t know if pinball really went that far. I don’t know how far Fenton went with them. They moved down to Chicago and I don’t know if Fenton went with them or not.

\textcolor{interviewer}{Interviewer:} […] I’m not 100% sure. Just to get a sense of the time span, do you recall what year you left the company?

\textcolor{interviewee}{Tom McHugh:} I left them when they were going down. The way I left them was, there were a couple people there that were working at Dave Nutting’s place that wanted to spin off their own organization that was called Action Graphics. Dave said, “Tom, we can’t go on like this, you should talk to this other guy,” his name was Bob Ogden who was running Action Graphics “I’m pretty sure you could work for him but we just can’t do anything anymore.” Because that was about the time the whole industry was going down.

That specifically was the thing that Dave Nutting was more into was the console games (what I mean by console games is these big wooden things). That’s what Dave was into and he was doing it for Midway manufacturing. At the same time Midway got into this other thing [Bally Professional Arcade] […] Something like that which Dave was getting into, but at that time the whole industry was going down. They weren’t selling anything and they couldn’t support employees. Action Graphics tried to take off and they did a couple things, they went under too then. I worked for them for a while until everything went bust.

\textcolor{interviewer}{Interviewer:} You spoke a bit to the relationship with Bally. What was your understanding of that at the beginning? Did they just say, “Hey, this is the people that we’re doing games for”. What did it feel like to you, the relationship between Dave Nutting Associates and Bally?

\textcolor{interviewee}{Tom McHugh:} Well maybe I should tell you one other thing first. When Dave Nutting said “We’re moving down to Arlington Heights” he says “You’re coming with us, aren’t you?” I said “No I’m not. That’s the last place I want to go is to Chicago.” This kind of went on for a month or two. 

There was a development system that Intel came up with called the NBS 800 which was the development system after the little blue box that they made which started the whole thing off. (They used to call that the Mod 80 or something like that). They had just gotten one in. I said, “Dave, I’ll tell you what, I will purchase one of these things if you want me to. I’ll work for you but I won’t go to Chicago.” So we thought it over and then he said, “Okay, let’s do that.”

At the time I lived over on the east side of Milwaukee with my wife and after a little while my wife and I decided we wanted to move out to Southwest Wisconsin, we wanted to get way out in the boonies. I worked for Dave Nutting in Milwaukee for a while, then we went out to Southwest Wisconsin and lived there for a while (still worked for Dave Nutting). That was about the time it may have switched over to Action Graphics. 

We decided after a while that both of our families were from Milwaukee and we were spending a lot of time going back and forth. We had the kids and it was like a 3 1/2 hour drive for Christmas, Easter, and stuff like that. We said, “We’re gonna move closer.” We ended up moving to a place called Plymouth, something like that, which is about an hour North of Milwaukee.

I worked for… I don’t know if it was Dave Nutting then or Action Graphics. It might have been Action Graphics, that might have been the change over time, just after we moved there. I was a ways from Arlington Heights for all this time. So my understanding of what Bally was doing and their connection with Dave Nutting was like third-hand.

\textcolor{interviewer}{Interviewer:} So when you were brought on was there actually a formal interview process? How did they gauge if you were any good at software to bring you on to DNA?

\textcolor{interviewee}{Tom McHugh:} There wasn’t any. They said, “We need people, we need them right now!”

\textcolor{interviewer}{Interviewer:} Okay. [Laughs]

\textcolor{interviewee}{Tom McHugh:} And Northouse said, “I know some people.” After doing that for a month or two, I went to Northouse and said “I want to work for Dave Nutting directly.” Basically I was working for Northouse who was contracting with Dave Nutting. He said, “Okay, if that’s what you want.” Really, that’s what I wanted. That’s what I did and started working for Dave Nutting directly.

\textcolor{interviewer}{Interviewer:} What sort of stuff were you doing through Northouse then?

\textcolor{interviewee}{Tom McHugh:} Oh, all kinds of stuff. Northouse had National Semiconductor [come] to him with a four-bit processor. They said, “You can have this 4-bit processor but you need someone… What we need from you- We’ll trade you this hardware if you give us some software.” Basically he wanted an assembler kind of thing, but when you get into four bits it’s just kind of weird. 

I was kind of the person, and then I was working as a teaching assistant. I went to one of the professors who was running one of the classes there, I said “Do you know anyone who would be interested in a final project?” He found two people who were interested in doing this thing, so I was kind of the overseer of these two people who were doing this thing for Northouse who was doing it for National Semiconductor. That was one of the things I did.

Another thing I did was on this MODCOM thing, they had a FORTRAN compiler. The way it worked, it basically converted FORTRAN statements into macro assembly statements and then compiling them (which is kind of a dorky way of doing things). We kept running into problems because one of the classes I had with Northouse I had to do some kind of frequency analysis or some crap like that. They kept coming up with different answers and I was going to the main school computer system and run something, then over to this MODCOMP to run the same thing; come up with two different answers. So another thing he had me doing was, “Go on and figure that out! Figure out what the problem is! Go talk to them and tell them to get their act together.” That was another thing.

Then I got involved with that robot for a while. They needed a way to talk to the MODCOMP, they needed to understand how the ports worked on whatever, then they had some sort of hose that went behind it or something. That was another thing I did. That was the kind of the thing- I was the only guy there so I did whatever really needed to be done.

\textcolor{interviewer}{Interviewer:} That exposure to the National Semiconductor chip, was that the first time you had seen a microprocessor or did you know about it beforehand?

\textcolor{interviewee}{Tom McHugh:} I think that was probably the first time. When we started doing that first project with Dave Nutting Associates, that was the Intel processor and I hadn’t even heard of that when we got there.

\textcolor{interviewer}{Interviewer:} Of the Intel processor, were you always using the 8080 or were you using the 4000 beforehand?  What were you first starting with?

[Side Note: The Intel 4000 Series is what’s meant here. There is no Intel 4000.]

\textcolor{interviewee}{Tom McHugh:} We started out with the 8080 because that was the first one. That’s all that was really available to do that. What Jeff built was a shifter kind of thing to trace different patterns because the patterns had to have different shifting whatever. That was done in hardware but it was done with 8 bits so he needed an 8-bit processor. That’s all that was available, there were no 16-bit processors.

\textcolor{interviewer}{Interviewer:} Was computer graphics ever an interest for you? Because of what you were doing was it more just making things move essentially?

\textcolor{interviewee}{Tom McHugh:} No. Neither, actually. I was more interested in the logic of something. Could be a computer graphics thing or could be anything. How do I make this thing happen, then what do you want to make happen? Then you go back and say, “This is what I want, how do I go do that?” So the software, it’s a design driven kind of thing.

\textcolor{interviewer}{Interviewer:} What were the systems that you were using? Were they the Intelect systems? I think that’s what they called the development kits for the microprocessors, the Intelect.

\textcolor{interviewee}{Tom McHugh:} Is that the thing that went into the MDS 800?There was something that plugged into it. On the one hand, let’s say you wanted to assemble something, you wanted to write some code and assemble that code so that it can be used somehow. That you would do with this MDS 800 thing. There also was something else that might have been on a separate board, it had this big ribbon cable coming off of it that at the very end looked just like the CPU. You plugged this thing into your board that you’re making up for the game (or whatever you’re doing) and that was the emulator. That might have been the Intelect, I don’t really remember what it was called. I think it was called an In-Circuit Emulator.

[Side Note: The Intelect was Intel’s original microprocessor development system, first released in 1973. Both the 4000 and 8000 series used 

\textcolor{interviewer}{Interviewer:} Were you loading things via paper tape or how were you loading it?

\textcolor{interviewee}{Tom McHugh:} Yeah, that’s how it started out. When we started out there was these machines- I think they were called KSRs (Keyboard Center Receive) – and it basically had a paper tape puncher on it and a paper tape reader on it plus this keyboard. What we started out doing, we had these pieces of paper tape that were your program. It was a real pain in the ass, because in order to make a change you have to re-punch the whole program out and then read it in.

\textcolor{interviewer}{Interviewer:} What was the first project then? What was the first thing you were told “Make some software for this” when you were in DNA?

\textcolor{interviewee}{Tom McHugh:} Yeah, it was a game called Gun Fight.

\textcolor{interviewer}{Interviewer:} There was nothing before that, just straight to Gun Fight?

\textcolor{interviewee}{Tom McHugh:} That was it, that was the very first one.

\textcolor{interviewer}{Interviewer:} Were you shown the game that they had imported, the Western shooting game? It was called Western Gun, I believe.

\textcolor{interviewee}{Tom McHugh:} No, I was not. Basically this is all Dave Nutting. Dave said, “Here’s what we want to do on the screen. We want to do this.” I went out and did that. “Then we want to do this.” I did that. I did the next thing. He showed me- I don’t think I ever saw that game. I may not have even been aware of it until much later.

[Side Note: According to Dave Nutting, he did not spend much time examining Taito’s Western Gun before communicating his game design ideas to McHugh. Western Gun was created by Tomohiro Nishikado, most famous for fathering Space Invaders, which in itself used a hardware similar to the one created by DNA.]

\textcolor{interviewer}{Interviewer:} So tell me a bit about the process of how you got to the end result. How were you able to make this puny, small microprocessor able to generate what was on the screen?

\textcolor{interviewee}{Tom McHugh:} Well really there was two parts to it: The background and foreground. There’s an interrupt, I think on this one, at the bottom of the screen. So at the bottom of the screen you’d do foreground stuff. In other words: You would write things onto the screen. I’m talking about the raster scan now, right? The raster comes down, hits the bottom of the screen, gets an interrupt, then start writing on the screen because the raster’s no longer there. You start writing at the top of the screen and what you don’t want to have happen is the raster catch up to you writing. So you do background stuff and then you do foreground stuff.

The foreground stuff was all immediate writing, I gotta do it right now. It may be that arm is moving and you have to move the arm to a new location. You might be shooting a bullet. That has to happen in foreground, but there are other things that happen too. There’s scoring and stuff like that happening in the background. That’s really the concept, you don’t really want to get caught in the raster. While the raster is away from you, you write stuff into that location on the screen.

\textcolor{interviewer}{Interviewer:} So the way that the arm moves, was that something where you were just drawing it directly to the screen or were you calling on something to tilt or signal another image? It had like six degrees of movement in front. Are you just literally, the character’s being redrawn with his arm in the new position or did you actually have something where you stored the arm movements like sprites? 

[Side Note: Discussed later in this interview are the difference between “sprites” and “objects” within a game development setting. Sprites are a particular type of object which can be positioned at any part of the screen, using it’s own separate memory. In these early games, objects had to be drawn with a bitmap and therefore were not independently rendered.]

\textcolor{interviewee}{Tom McHugh:} Yeah, I think the arm movements were stored. I think there were two parts to that because there was like a mirrored reflection down the center of the screen, you look at the left and the right. The right might have been the mirror on the left, that kind of thing. So I think the original positions were stored in memory as specific things but the mirror image was calculated in the background and put into an area of memory that wasn’t on the screen at all. That’s where it was used from, but it had to be reversed. It wasn’t the full image, it was just whatever changed.

\textcolor{interviewer}{Interviewer:} So that would be like the movements of the arms and the feet and that’s pretty much it?

\textcolor{interviewee}{Tom McHugh:} Yeah. This is real primitive stuff here because that processor couldn’t handle a whole lot.

\textcolor{interviewer}{Interviewer:} [Laughs] Right. How did you insert the things then? I’ve heard a lot of ways that people implemented graphics. Was it, did you have to do the whole hexcode? You had to calculate it on graph paper? How did you get it in?

\textcolor{interviewee}{Tom McHugh:} Yes, yes. That’s how you got it in. It was probably in hex. Basically Dave Nutting would draw it out, I’d translate into hex, and translate the hex into a table that went into the EPROM (well actually the ROM). Everything we did was with EPROMs. Jeff came up with the blacklight. 

The concept is an EPROM looks just like A CPU chip except it’s got this little window in it, right in the center. You take the EPROM, set it right on the blacklight for like half an hour and that will erase it. That was the thing we did to erase them, I think. Then we would run that program into the EPROM and that’s how it was executed.

\textcolor{interviewer}{Interviewer:} Did you ever have a situation where you only had a partially erased EPROM or was an hour pretty much good to wipe it clean?

\textcolor{interviewee}{Tom McHugh:} Yeah, it was pretty much good enough. It wasn’t a big deal. You just went and got another, there was like five or six of them sitting there.

\textcolor{interviewer}{Interviewer:} How did you add in the music?

[Side Note: As it turns out, it’s an error to assume that Gun Fight had music. Due to some improper emulation, the famous “Funeral March” by Frédéric Chopin was edited into some versions of the ROM online. The sequel to Gun Fight, Boot Hill, does have music. See a blogpost by this author for further reading. https://thehistoryofhowweplay.wordpress.com/2019/06/13/first-music-the-mystery-of-the-first-video-game-soundtrack/  ]

\textcolor{interviewee}{Tom McHugh:} That was the one thing that I didn’t do. First of all, you have to understand that it’s not music. It’s sounds. What happened was, someone would say “This is the kind of sound we want” and somebody else like Dwyane maybe would start putting probes on different parts of the hardware and saying “Does this give us the frequency for the sound that we’re looking for?” So the sounds basically came off the board. That was early on.

When we started getting into actual games with these custom processors I think I wrote a little sound generator kind of thing and there was another guy who did all the sounds. I would do all the work on the game, he would do the sounds. I’d say, “We need this sound, blah blah blah, here it is.” The processor was already on the chip. He would give me- It’s almost the same thing we were talking about a little while ago, just a table of sounds. That was later on, so you have to understand the distinction here between the early days and how things are really primitive and the later days where things were less primitive (but they were still primitive).

\textcolor{interviewer}{Interviewer:} [Laughs] Who did you mention in terms of the sound? I think you said a name and I don’t think I caught it.

\textcolor{interviewee}{Tom McHugh:} No, I did not say a name and I can’t remember the guy’s name. If you could rattle off a bunch of names I could remember it but I can’t remember his name.

\textcolor{interviewer}{Interviewer:} I’m not sure who worked [there] early on. All I know it was Fenton, Nutting, Fredricksen, and you. That’s the only people that I know. 

[Side Note: Dwayne Knudston and David Otto were there at the start of DNA though at this point it’s unclear if they were actual employees.]

\textcolor{interviewee}{Tom McHugh:} This was after they moved down to Arlington Heights. It was a guy that they hired then and he would do other things down there too. He was kind of like an interface to Dave Nutting Associates, they did a lot of things just for me. For example, one of the things that we ended up doing was we were using a modem to get the code for the game down and he would be on the other end of the modem.

\textcolor{interviewer}{Interviewer:} Do you have any other stories about creating Gun Fight then? Was there any complications or any interesting parts of it?

\textcolor{interviewee}{Tom McHugh:} There was one complication, Jeff Fredricksen got mad at me once. He said, “You know, I’m going to take over the foreground part of this.” So that’s what he did. He did the foreground stuff. The problems were you had to be very careful where the raster is. The problem is just a bullet is really two pixels high and the pixel here is a really large, massive thing as opposed to what you see today. If you don’t get it right, what you see is basically the old pixel and the new pixel but they’re not next to each other because they’re one raster removed from each other. There’s stuff like that you have to worry about.

\textcolor{interviewer}{Interviewer:} As far as you know then, was that the first video game that ever used a microprocessor in like any capacity?

\textcolor{interviewee}{Tom McHugh:} I would say yes.

\textcolor{interviewer}{Interviewer:} There’s been a lot of people that have been trying to definitively say whether or not it is. We’re pretty sure, we’re just not 100%.

[Side Note: For further reading on the early microprocessor contenders, see this blog post by the author.] https://thehistoryofhowweplay.wordpress.com/2018/09/11/microprocessors/ 

\textcolor{interviewee}{Tom McHugh:} Yeah. You see the problem is Intel just came out with that, at that time, so there really wasn’t time to do something before that. The only other thing that was graphically oriented at the time was Pong which was not really done with a microprocessor.

\textcolor{interviewer}{Interviewer:} Was the game ever known by anything else other than Gun Fight?

\textcolor{interviewee}{Tom McHugh:} No. Not that I’m aware of. That’s what was on the screen.

\textcolor{interviewer}{Interviewer:} [Laughs] You had to write that whole thing out? Did you create a character set in the thing? You made a ‘G’ and a ‘U’, things like that? Did you make those an interchangeable thing or did you just-

\textcolor{interviewee}{Tom McHugh:} No, they were just a table. A table of some bytes.

\textcolor{interviewer}{Interviewer:} When you were creating the game did you know it was going to be a hardware system that was going to be carried forward or was it entirely “We’re making this product now”.

\textcolor{interviewee}{Tom McHugh:} That’s difficult. I assumed it was gonna move forward because it made so much sense. All they had to do was make a profit so they could do the next one and I couldn’t see them not making a profit. It didn’t have to be a huge profit, but I think on the other hand they were doing this for Midway under contract so they were getting some money for it that they could count on. There was more of a long-range relationship there.

\textcolor{interviewer}{Interviewer:} As far as you know, was it a good success?

\textcolor{interviewee}{Tom McHugh:} Yeah! Yes. In the sense like Pong was. I mean Pong took the world by storm but not a lot of people really noticed. You have this kind of dichotomy thing that people really into that thing said, “Hey, this is really neat!” but most people weren’t into that thing, “What does that mean?” It wasn’t until the home versions of things started appearing that people started noticing (I’m not sure when that happened, to tell you the truth).

\textcolor{interviewer}{Interviewer:} After Gun Fight, was Sea Wolf right after that?

\textcolor{interviewee}{Tom McHugh:} Yes, yes.

\textcolor{interviewer}{Interviewer:} What was sort of the inception of that then?

\textcolor{interviewee}{Tom McHugh:} That was Dave Nutting. Dave said, “Here’s what we want to do.”

\textcolor{interviewer}{Interviewer:} Okay. [Laughs] So with all these games you were just coming from a complete blank slate? You were just interested in the technical side? You didn’t really know how people had done it before at all?

\textcolor{interviewee}{Tom McHugh:} Oh no, I had no idea how people had done it before. In a sense Dave would say, “This is what we want to do” but then I would take that and make it into a real game. It wasn’t that Dave had everything all set up, he would present an overview of “Here’s how this is going to happen.” 

We did this one game called Wizard of Wor and basically he said “We want to do something like this. What do you think? How do you think we should do this?” Basically he wanted to do this and I’d put it all together to make it into a game, that kind of thing.

Kind of the same thing with Sea Wolf and kind of the same thing with Gun Fight except Gun Fight there was so few things you could possible do with it that Dave said “We want to do this and this and this” and that’s it! It wasn’t all Dave saying “Do this” and me saying and me doing it, it was kind of a transfer of the design or logic of what you want to do.

\textcolor{interviewer}{Interviewer:} To speak one more thing on Gun Fight, say the fact that you had limited ammo, was that something that came completely from Dave? “That’s what you need to do”?

\textcolor{interviewee}{Tom McHugh:} Yeah, probably. Yeah. That would probably be coming from Dave.

\textcolor{interviewer}{Interviewer:} Mhm. On a technical sense that means you need to put a little bullet icon on the screen and then remove it. That’s something he would tell you, that “We need this bullet icon and it needs to disappear”?

\textcolor{interviewee}{Tom McHugh:} Yes. Until you mentioned that, I didn’t remember that it even did that.

\textcolor{interviewer}{Interviewer:} Yeah. Cuz that’s one of the things that’s different from the original, the Western Gun game, that you have limited ammo. That’s something that he came up with then.

\textcolor{interviewee}{Tom McHugh:} Yeah.

\textcolor{interviewer}{Interviewer:} Did the theme of Sea Wolf seem kind of weird? You’re jumping from a Western game to a submarine game and then later a baseball game? Did that seem kind of weird?

\textcolor{interviewee}{Tom McHugh:} No, not at all. It’s basically, you have a blank palette and you’re going to do something with it. It doesn’t have to be the same thing as you last did. As a matter of fact you probably want to do something different to appeal to different kinds of people. There was no interlude in between them. It was like, “Okay, we finished this, we’re going to start the next one.” We’re not going to sit around for six months and see what happens.

\textcolor{interviewer}{Interviewer:} How long generally did it take to complete a game? Those early games, how long did those take?

\textcolor{interviewee}{Tom McHugh:} I’d like to say three or four months. I never really paid much attention to it though so I don’t know if that’s accurate or not, but that’s about what it seemed like.

\textcolor{interviewer}{Interviewer:} Right. Did the hardware significantly change at all between Gun Fight and Sea Wolf?

\textcolor{interviewee}{Tom McHugh:} No.

\textcolor{interviewer}{Interviewer:} There seems to be a lot more objects on the screen in Sea Wolf so was that just optimization?

\textcolor{interviewee}{Tom McHugh:} I can’t answer that. I don’t know. The problem is that there were two Sea Wolfs and the second one sticks in my mind more than the first one but the second one came way later.

\textcolor{interviewer}{Interviewer:} Like two years, thereabouts. One question, maybe you might be able to answer, I don’t know if it’s also a technical thing or it’s just the way you were told… The enemies in Sea Wolf, they don’t shoot back unlike in later games. Do you have any idea why that was?

\textcolor{interviewee}{Tom McHugh:} That’s what Dave wanted. He said, “We’re trying to increase score based on the three different sizes of the things that you hit.” You get more based on smaller things that are moving faster. It’s basically just a scoring game.

\textcolor{interviewer}{Interviewer:} So you think you probably could have if you wanted to have them raining down bullets as well, Space Invaders style?

\textcolor{interviewee}{Tom McHugh:} No no. No, because there’s no way to feed that back into the game itself. It didn’t fit the design and that’s what he wanted. He wanted something that basically just moving across the screen.

\textcolor{interviewer}{Interviewer:} Right, but in a technical way you could have had those things coming down.

\textcolor{interviewee}{Tom McHugh:} Oh yeah, but there’s a whole bunch of definitions that have to be put in place for that. You know, what does it mean to have something coming back at you?

\textcolor{interviewer}{Interviewer:} I imagine that sort of creates clutter, in terms of visuals. I don’t know how many objects you could actually drive on screen with that hardware.

\textcolor{interviewee}{Tom McHugh:} Yes. At some point what they did is they changed from the 8080 to the Z80 and I don’t remember when that was.

\textcolor{interviewer}{Interviewer:} Was that when they went color? The original games, they were monochrome and then later you had a color system. Was the Z80 the color system?

\textcolor{interviewee}{Tom McHugh:} Well the thing is the Z80 didn’t really have a whole lot more than the 8080 did but it had the ability to do certain things that the 8080 did not have. It was just like an addon to the 8080, it wasn’t a complete “Hey, this is like 10 times better.”

\textcolor{interviewer}{Interviewer:} So after Sea Wolf was Tornado Baseball?

\textcolor{interviewee}{Tom McHugh:} Yes.

\textcolor{interviewer}{Interviewer:} Did you guys ever do little experiments of games that didn’t go into production anywhere or were the things that you were put on pretty much “This is the stuff we’re going forward with”?

\textcolor{interviewee}{Tom McHugh:} Actually there was maybe only one thing I can think of and that was many years later. After we did Wizard of Wor, Dave wanted to do a variation of Wizard of Wor […] [to] walk in space or something like that. He said, “We want to do this, we want to do that. Can you do it?” I said “Yes but I doesn’t make any sense. Yes but I doesn’t make any sense.” [Laughs] So I did it, it didn’t make any sense, and it didn’t go anywhere. Visually there’s nothing to hang your hat on there. The cadet doing something else and there was never the rest of that designed. I didn’t know where he was going with it, but it was like “Well, yeah we want to do… Something.” That was it.

\textcolor{interviewer}{Interviewer:} I know Fenton said the first thing that he was working on when he came in was a poker game of some kind. Were you guys working together at all or were you just entirely in separate corners?

\textcolor{interviewee}{Tom McHugh:} Actually we were almost completely separate. I’m not even sure what that poker game was. With one exception though: At the very end of the Sea Wolf game, I caught something. I had the flu or something like that or influenza. I was out for a week right at the very end of the game. The game was all done and the only thing that had to be added to it was the settings. I hadn’t done that, I caught something. So Fenton came in and he would work on that. He would give me a call at home, I’d get out of bed, I’d stay on the phone for a few minutes, I’d hang up, and get back in bed. He would go work on the rest of it, he finished up Sea Wolf.

\textcolor{interviewer}{Interviewer:} Good thing too because it was also very successful.

\textcolor{interviewee}{Tom McHugh:} Yeah. Well it would just have waited another week, that’s all.

\textcolor{interviewer}{Interviewer:} Well I don’t know. Were you under strict deadlines at all?

\textcolor{interviewee}{Tom McHugh:} Not really, no. In the sense that Dave would say “How you doing?” I’d say “Well, I think here’s where we are.” He would work out the deadlines with Midway, but I never really was part of that. There was never a time that we said something would be available (with the possible exception of Sea Wolf that I just talked about) at a certain date and it wasn’t.

\textcolor{interviewer}{Interviewer:} Right. The baseball game, I had heard that I think Fredricksen was working on it before it actually got going. Was there anything there or was that also from scratch?

[Side Note: That the baseball game was the first video game being worked on at DNA was claimed to Keith Smith of All In Color for a Quarter by David Nutting. This may not be true based on other testimony about the early development system by Jeff Fredricksen.]

\textcolor{interviewee}{Tom McHugh:} That was from scratch as far as I know, other than to say when you have this different kinds of things you have to have an idea of what the layout’s gonna be. How is the screen going to be oriented for the front panel, for example, where the controls are? What are you seeing? If you’re gonna see a ball park, then how’s that going to work? For example, there might be a mirror in there and the screen might be laying down and facing up. The mirror reflects it to you so everything you have to do is reversed except that the reversal might be done in hardware, so you don’t really care about reversing it from a software perspective.

\textcolor{interviewer}{Interviewer:} Who was it that came up with the mirror then? It’s really cool to have that color background even though it’s not really in color.

[Side Note: Using mirrors was a well-trod method to display fancy visual effects in arcade games. David Nutting appears to have been the first to apply it to video games however, using it to give the monochrome image of Tornado Baseball a colored background which did not bleed onto the game objects.]

\textcolor{interviewee}{Tom McHugh:} I don’t know. That’s Dave Nutting and wherever he got that from. Midway had done a lot of these things because that’s what they did, built these arcade games.

\textcolor{interviewer}{Interviewer:} Right. Talking about the number of objects on screen, I do find it kind of funny that in that game you have all the outfielders represented but the one person who isn’t represented is the batter. You just have a little line that does the bat. Was that a thing that you were told to do or was that a technical thing?

\textcolor{interviewee}{Tom McHugh:} Don’t know. Those things really aren’t that important. What’s important is you have these joystick controls or whatever, maybe it’s a button to- I don’t remember how the ball came out. Something caused the ball to come out then you had this joystick control that controls where the bat is actually swinging and when it’s swinging. So the ball and the bat are the important part of the whole thing. Does the bat hit the ball correctly.

\textcolor{interviewer}{Interviewer:} [Laughs] Since all those games were two players, was that ever a complication? Were there concessions that had to be made for that? There’s only one screen so it’s not like you’re doing split screen or anything.

\textcolor{interviewee}{Tom McHugh:} No, I mean you couldn’t. You couldn’t not do it. For example, Gun Fight… Was it two people or was it one person and a computer?

\textcolor{interviewer}{Interviewer:} No, it was two. Maybe Boot Hill has AI, I’m not sure.

\textcolor{interviewee}{Tom McHugh:} Yeah, I don’t remember.

\textcolor{interviewer}{Interviewer:} Was it, you were spending all your time driving graphics on the screen or was it that nobody thought you could do an AI opponent at all?

\textcolor{interviewee}{Tom McHugh:} You couldn’t. There wasn’t enough time. These were really primitive chips.

\textcolor{interviewer}{Interviewer:} One other thing with the baseball game, do you ever recall having to make a fix for it? There’s this program card mod that needed to be slotted into the machine. I know that’s hardware, but do you ever recall that?

\textcolor{interviewee}{Tom McHugh:} No, not at all. I don’t know, did you ever find if that game… Was it a good game or did it bomb? [Laughs]

\textcolor{interviewer}{Interviewer:} [Laughs] It sold less than Sea Wolf and Gun Fight but Tornado Baseball was pretty good. All three of those games got sequels so I imagine they were all good performers. [Laughs]

\textcolor{interviewee}{Tom McHugh:} I always thought Tornado Baseball was really dumb. [Laughs] That was my opinion.

\textcolor{interviewer}{Interviewer:} [Laughs] What was your thought on it then?

\textcolor{interviewee}{Tom McHugh:} Oh, it’s because it wasn’t intuitive what you had to do win the game. It was very complicated. The distinction between hitting a good ball and missing the ball wasn’t all that obvious, whereas with Sea Wolf you could shoot something, you could see it going up and see “Oh, it’s missing!” You could tell there’s something going across the screen at a certain speed and this thing is going up vertically at another speed and you could use that to reevaluate how you’re going to do things.

\textcolor{interviewer}{Interviewer:} So were you starting to get a sense of how players were going to react to the games? Obviously you guys weren’t doing market research or anything. Were you starting to get an intuitive sense of how to make systems that were easy to interact with?

\textcolor{interviewee}{Tom McHugh:} No. No, it was more like “what appeals to me” or “what does Dave want to do” and try to make it interesting to me. Sometimes you can do that and sometimes you can’t.

\textcolor{interviewer}{Interviewer:} [Laughs]. Yeah. Fickle bosses.

\textcolor{interviewee}{Tom McHugh:} Yeah, you look at it and say “This is really dumb and it’s hard to do” so you don’t want to do that. You want to do something that makes more sense. Just a bit of common sense there.

\textcolor{interviewer}{Interviewer:} One last thing did you have any involvement with the consumer products division stuff they were doing? You mentioned the console before. I know they also did a pinball table and things like that.
\textcolor{interviewee}{Tom McHugh:} I did one game. They did this – I forget what it was called – tabletop type game, you put a cassette into it. I did one of those. The name of the game was called Cosmic Avenger. We were talking about this guy whose name I couldn’t remember who did the sounds, apparently he was trying to do a game, this game called Cosmic Avenger. He got to a certain point and said, “You know, I just can’t do it. It’s not working out. Tom, could you do it?” I looked at it and said, “You know I’ve never really done anything like this, but sure, let’s give it a try.” So we did that and we did Wizard of Wor I think for that one too. Wizard of Wor was an arcade game then it came out on this other thing.

\textcolor{interviewer}{Interviewer:} But you didn’t have any involvement when Bally was creating that console system or anything like that? You were just solely doing arcade games through that whole period.

\textcolor{interviewee}{Tom McHugh:} Pretty much, yes. Midway was the one who did those, who went to the other way too. I went down there, and I got to their factory, they were showing me around at the time they first started building those things. They showed me how you lay out the circuit boards and all that equipment for that kind of thing.

\textcolor{interviewer}{Interviewer:} And you just generally weren’t interested in that because you were mainly a software guy?

\textcolor{interviewee}{Tom McHugh:} Well it was interesting but I really was more into the arcade thing more than anything else. It really was a different kind of environment but not that different since Jeff came up with this custom chip and I think it was used both in the arcade games and the home games. I think it was the same chip. So the distinction between the two wasn’t all that great, but on the other hand there are things you have to do in an arcade game that you don’t do in an arcade game and vice versa.

\textcolor{interviewer}{Interviewer:} So it couldn’t be directly ported over or anything.

\textcolor{interviewee}{Tom McHugh:} Correct.

\textcolor{interviewer}{Interviewer:} I imagine a lot of that was also the pixel size that you talked about before. I think the system that you guys had was something like 256 by 222 and home games were much different than that.

\textcolor{interviewee}{Tom McHugh:} That could be, yeah. I’m not sure about that though. 

Last I talked to Dave Nutting was when he said, “You should go work for these other people”, the people I told you about, Action Graphics. That was it. Never talked to him after that. All the time I of course had not been working there. I’d go down to Arlington Heights a couple times a year and that’s it. At the time I might have lived in Southwest Wisconsin. It wasn’t a daily kind of thing, seeing him daily and finding out what’s going on, it was just really hearing from other people.

\textcolor{interviewer}{Interviewer:} I wasn’t 100% clear on the arrangement that was going on. So how were you working with them? You didn’t go down to Arlington. Were you just calling over the phone and coding some stuff? What was going on there?

\textcolor{interviewee}{Tom McHugh:} When I went down there I would basically go down there to talk to either talk things over or pick up some hardware for our new games. It wasn’t very often. I was basically just an employee of this, like any other employee. I saw the thing that Fenton did and I’m not sure what his relationship with DNA was. Initially he was an employee and there were times that I saw him down there and he’d be working on one thing or another thing. I thought he was an employee at the time, but seeing that video sounded like he really wasn’t an employee, he was kind of on a stipend or something like that. I don’t know. I really don’t know. All the time that I was there I was basically an employee, but I was not working there, I was working remotely.

\textcolor{interviewer}{Interviewer:} And did you ever have an official title there?

\textcolor{interviewee}{Tom McHugh:} No. Not at all. I wasn’t aware that anybody had any titles.

\textcolor{interviewer}{Interviewer:} Yeah. [Laughs] That’s the sense that I got, that Dave wasn’t very much a guy for hierarchy or anything like that.

\textcolor{interviewee}{Tom McHugh:} Yeah, okay. We’re all just doing a job. Addressing things that were interesting and getting paid for it.

\textcolor{interviewer}{Interviewer:} Yeah. It didn’t seem to me that David saw himself as a President or anything. It’s just, “This is Dave Nutting Associates, we’re doing games for Bally.”

\textcolor{interviewee}{Tom McHugh:} Right. I never got that impression either.

\textcolor{interviewer}{Interviewer:} You said initially you were working in the warehouse. Did they ever get a real facility before they moved down to Arlington?

\textcolor{interviewee}{Tom McHugh:} No. They went right from the warehouse down to Arlington Heights.

\textcolor{interviewer}{Interviewer:} Do you recall what game you were working on when they moved? 

\textcolor{interviewee}{Tom McHugh:} No. [Laughs] A lot of that is really fuzzy. The one video that we were talking about, the early Wizard of Wor game, it kinda sort of almost jogged a memory but I couldn’t exactly remember how all that went. [Note: In between the two hone calls I emailed the video to Mr. McHugh to see what he could remember.) It looked like that first one was almost a prototype and there was only one of them. Is that possible, that there was only one of those things?

[Side note: Dave Nutting Associates moved from Milwaukee to Arlington Heights, Illinois in 1977, which was prior to Wizard of Wor’s development. It’s unclear what McHugh would have been doing between his early games and Wizard of Wor but he does claim to have worked on a number of clone games for Bally in the late 70s period.

McHugh refers to a video of an early version of Wizard of Wor called Invisible Monsters, uploaded by Youtube user Crimefighter. https://www.youtube.com/watch?v=DGzRTOmd3H4 ]

\textcolor{interviewer}{Interviewer:} Yeah. To my understanding, normally what you do is once you guys got into software you would make a version of the game, put it on location, see if it makes money, and then go back and finish it basically. Is that how you recall?

\textcolor{interviewee}{Tom McHugh:} No, not really. Again, I’m trying to remember how that all went, but I’m so away from it because I was remote. That could be possible that they could put one out there for a while to see how it does.

\textcolor{interviewer}{Interviewer:} And then just forget to take it back? [Laughs] Essentially.

\textcolor{interviewee}{Tom McHugh:} Yeah. Depends on what you mean by ‘put it out there’. Is it one of the employees who owns a string of arcades, is he taking it out? Or are they actually selling it to somebody? I don’t know.

\textcolor{interviewer}{Interviewer:} I have no idea where the thing came from, just suddenly someone posted it on Twitter. I was like, “Oh, that’s kind of interesting. I was trying to contact this guy.” [Laughs]

\textcolor{interviewee}{Tom McHugh:} Yeah. But just one of them? Did they leave a small amount? I don’t know, I can’t remember that.

\textcolor{interviewer}{Interviewer:} Right. Do you recall that being the name of the title in development though, Invisible Monsters?

\textcolor{interviewee}{Tom McHugh:} I can’t remember that either. I really really don’t remember that. The thing of it is, you asked me what they started working on when they moved down to Arlington Heights. That may have been the game right there. The start of it.

\textcolor{interviewer}{Interviewer:} Right. That would have been around the ‘80 period, somewhere in there.

\textcolor{interviewee}{Tom McHugh:} Yeah.

\textcolor{interviewer}{Interviewer:} You talked about Bob Ogden before. Can you tell me a bit about him, working with him there and later onat Action Graphics?

\textcolor{interviewee}{Tom McHugh:} Yeah, I don’t remember if Bob Ogden was even a part of DNA. I really don’t remember that at all. My earliest memory of him is Action Graphics, but there had to be something there. There had to be a very close connection somehow.

\textcolor{interviewer}{Interviewer:} He claimed to have helped on Wizard of Wor a little bit but that might have been from the Arlington location helping Dave shape what was going on.

\textcolor{interviewee}{Tom McHugh:} Okay. The games that I worked on, I was the sole person on the game part and usually there was a guy who did the sounds. There really wasn’t a lot of collaboration when you were doing a game, it was just the one person doing it, otther than the feedback from Dave Nutting (or the direction from Dave Nutting [Laughs], one or the other).


\textcolor{interviewer}{Interviewer:} Do you remember Scot Norris at DNA?

\textcolor{interviewee}{Tom McHugh:} Yes. He was the person I was talking about before.

\textcolor{interviewer}{Interviewer:} Okay. He was the sound guy.

\textcolor{interviewee}{Tom McHugh:} Yeah. I don’t know what his role was. He was kind of the other end. He did the sounds and he did basically… I don’t know!

\textcolor{interviewer}{Interviewer:} Before I was saying that they didn’t really departmentalize things, but it was kind of you in one section then Fenton in another section doing the consumer stuff, right? You didn’t really get involved with that.

\textcolor{interviewee}{Tom McHugh:} Initially, like I said, Fenton came into do the pinball thing. It was more like a project kind of thing. That’s how the breakout [came and went]. I was more familiar with the 8080 and Z80 stuff, that’s why I stayed along with that. I think the reason Fenton eventually got into games themselves was because there was a need for more people to do games as opposed to a need for more people to do pinballs. That’s how he got into it. T

he processors themselves were so much different that there was a little bit of a learning curve there. You don’t say, “Today we’ll do this and tomorrow we’ll do that other one which is really dissimilar”.

\textcolor{interviewer}{Interviewer:} [Laughs] Right. I know he was put on some pretty challenging stuff when he got that. He had a first person driving game, a card game sort of thing, Checkmate. He got thrown all of that stuff and you kind of got the sequels. Sea Wolf II, Boot Hill, Extra Inning, all that stuff.

\textcolor{interviewee}{Tom McHugh:} I don’t think I ever got into the sequels.

\textcolor{interviewer}{Interviewer:} Oh okay.

\textcolor{interviewee}{Tom McHugh:} I did a first person driving game once too.

\textcolor{interviewer}{Interviewer:} Oh, okay. That was the Midnite Racer thing?

[Side Note: Jamie Fenton also claims to have done Midnite Racer/280 ZZap as a solo project.]

\textcolor{interviewee}{Tom McHugh:} Yeah, that might have been it.

\textcolor{interviewer}{Interviewer:} Do you recall anything about how that idea came up?

\textcolor{interviewee}{Tom McHugh:} No. That’s just another Dave Nutting, “Here’s what we want to do now.” They’d just look around and see what games are popular. Driving games were popular and shoot ‘em up games were popular so that’s kind of where these came from. “What’s popular? Let’s do our own variation of it.” That Gun Fight game was really somebody else’s game but we did our version of it. It wasn’t like you buy their game then put their game into a console and sell that.

\textcolor{interviewer}{Interviewer:} I have heard that Dave kind of separated the people who were doing game design and the engineers, in terms of bringing up ideas for games. You said in terms of people you were interacting with that was mostly Dave. Were there other people that tried to ascend to that level or were they mostly happy being engineering staff?

\textcolor{interviewee}{Tom McHugh:} See, that’s just it, I didn’t really interact with a lot of people so I don’t know. It was just Dave and that was about it. After a while the only reason I talked to Jeff is he made this computer we were using after a while, we called it the ICE box. It was the next thing after the MDS 800 after that turned out to be- Actually I think the MDS 800 was really just for the 8080 stuff and then when we went to the Z80 is when Jeff made a bunch of these boxes and called them the ICE boxes.

They were computers that were maybe three feet long, eight inches high, and eight inches deep. About as high and deep as a standard card was at the time. That’s probably why it was designed the way it was, because the hardware people designed it for the hardware cards they put in. That was what we used and that was what I used right up until the end.

There was a thing you could get on the telephone line you’d get spikes on there, feels like lightning or something. It would blow out something on one of the boards which would make the whole thing inoperable. If there was a lightning storm and it hit one of the electrical lines and brought it into the house. This happened maybe once or twice a year. I’d have to go down and get it fixes. [Laughs] That was probably the sum total of my interaction with Jeff.

\textcolor{interviewer}{Interviewer:} Did the ways of transferring software get any more sophisticated when you moved to the Z80 system? Before you were saying it was paper tape and typewriters. Were the ICE boxes any more sophisticated?

\textcolor{interviewee}{Tom McHugh:} Well, okay. The first thing we started out with was the paper tape but the next thing we went to was these big 8 inch platters of floppy disk. That might have been in the first game, even. We were doing the paper tape stuff for a little while but not for real long because that was just really difficult to work with! The company name that we used was ICOM. They made these boxes that had four floppy disks in them, 8 inch floppies, but typically the boxes we had only had two floppy drives and that’s all we really needed.

I remember, just after they left (I was still in Milwaukee with my wife before we moved out to Southwest Wisconsin), I contacted the ICOM people and I said “Hey, would you be interested in an assembler for this? Something that you could distribute when you sell these boxes.” They said, “Yes, how much do you want?” I said “\$10,000.” They said, “That’s too much.” So I said, “What about if I make it for both the 8080 and the Z80?” and they said, “Okay, we’ll do that.” [Laughs] So I made an assembler to do that to distribute with the box that they sold.

Then I thought, since I was using the assembler myself to put things together for the games, I actually made a little operating system for it. It’s kind of funny because the way the operating system worked (it was really primitive), it worked with two floppy drives. One floppy drive you would do all your editing on and the other you would do your assembling and linking on. That would be the second one. I used that for years. They wanted to use this other operating system called CP/M. That was one of the early on ones but it had some problems because of the areas of memory it wanted to use conflicted with what we wanted. That’s why I ended up writing my own operating system.

Years later the first hard disk came out. This is really cool. They sent me this hard disk but it came in this big metal box. The box was like, oh, 16 inches wide, 12 inches high, and 12 inches deep, something like that. It was all sheet metal. That was the box. Then on one side of it there’s a ribbon cable connector and a power cable connector. I got really interested, “What is inside this box?!” 

I went and got a screwdriver to open up the box and inside the box was one of the first generation hard disks, which is actually the same size as the second generation hard disk in it’s footprint but it was maybe three inches high. When you looked at this it’s this semi little thing that’s sitting in the middle of this big, gigantic box and I thought, “Why did they do that?” [Laughs]

I ended up adapting my operating system to do the same thing. I ended up having a whole bunch of drives that went up to ‘X’ I think. [Laughs] X drives! They were all the same size as some of those 8 inch floppy drives and they all worked the same way. You had one for editing and the other one for compiling. I ended up using that for the rest of the time that I was there.

\textcolor{interviewer}{Interviewer:} Did you ever make any debugging tools for it?

\textcolor{interviewee}{Tom McHugh:} No. Initially the debugging tools- I talked about this circuit emulator thing. It was a board that went into the MDS 800 and it was the emulator for whatever your hardware was. You’ve got a motherboard and it’s got a CPU socket and instead of having the CPU in the socket you have this emulator plugged into that socket. You would run through this board, it would emulate whatever you built.

The first generation we used the EPROM and I told you about the blacklight and how we erased them. This is the second generation. After the EPROMs were gone we were using this emulator. The third generation was this ICE box thing that I just talked about, this low, wide computer thing. That’s what Jeff was doing, making all that hardware up himself and then implementing it.

\textcolor{interviewer}{Interviewer:} Did you have a terminal for it or were you still using a teletype?

\textcolor{interviewee}{Tom McHugh:} Oh no, we had a terminal. It was an ADDS Consul 580 or something like that. You see some of those old shows with the first generation CRTs that look kind of look really [modernistic]? It was one of those. Big, huge sucker.

\textcolor{interviewer}{Interviewer:} […] I want to see if you recall this. Do you remember seeing Space Invaders for the first time?

\textcolor{interviewee}{Tom McHugh:} First time I saw Space Invaders was in an arcade. I thought, “That’s pretty cool!”

\textcolor{interviewer}{Interviewer:} Did you frequent arcades or was that just kind of a thing to do?

\textcolor{interviewee}{Tom McHugh:} I did not frequent them, but if I was near one I would check it out just to see what was popular.

\textcolor{interviewer}{Interviewer:} And hope that people were crowded around the Midway games? [Laughs]

\textcolor{interviewee}{Tom McHugh:} Yeah. [Laughs] Whatever. I still do that. If I’m near one, just to see what people are putting money into.

\textcolor{interviewer}{Interviewer:} Right. That’s just something you’ve always kept up on, to see what technology is the flavor of the day?

\textcolor{interviewee}{Tom McHugh:} Yeah. Other than that I really didn’t go in to play.

\textcolor{interviewer}{Interviewer:} So with that sort of game coming out and with that new hardware system that’s all in color and everything, was there any pressure to really up the fidelity of games? To make the humans look more human and make more sprites moving on the screen?

\textcolor{interviewee}{Tom McHugh:} Yeah, definitely, but there’s always a limitation. “What can you do?” I mean it’s very much a shoehorn kind of thing because these systems are really primitive. How can you make the things you want to do get done with the hardware that you have? It’s not like you have a blank slate, you can do anything you want. It was the opposite of that. “Here’s what we’d like to do. Can you even do that?”

\textcolor{interviewer}{Interviewer:} You said at Action Graphics you did home stuff, right?

\textcolor{interviewee}{Tom McHugh:} Yes.

\textcolor{interviewer}{Interviewer:} Did that feel restrictive, having to bend to that hardware while still upping what the games could do?

\textcolor{interviewee}{Tom McHugh:} Not really because those systems were pretty much the same as the console systems. As far as the availability of the hardware they’re pretty much the same. They weren’t better, per say. They might have been a little bit more modern in the sense that ten years ago things were a little bit more primitive no matter where you were. In the home, in the consumer side, or in the commercial side, but that’s normal you know.

\textcolor{interviewer}{Interviewer:} Do you recall a game that was worked on at DNA called “Demons and Dragons”?

\textcolor{interviewee}{Tom McHugh:} No.

\textcolor{interviewer}{Interviewer:} Did you ever work on any of the vector games there?

[Side Note: DNA prototyped a number of vector games as well as a Tron arcade game which were never put into production.]

\textcolor{interviewee}{Tom McHugh:} No. I remember being shown the vector games but I never worked on one.

\textcolor{interviewer}{Interviewer:} Was that just because you were headed out at that time, you didn’t want to work on it, or you were just never given the prerogative to work on it?

\textcolor{interviewee}{Tom McHugh:} I don’t know. I would guess though that the hardware and the development hardware was different, but that’s just a guess.

\textcolor{interviewer}{Interviewer:} I think after that was Wizard of Wor. What was the idea that Dave came in with?

\textcolor{interviewee}{Tom McHugh:} I don’t remember that. The problem with Wizard of Wor was there was several variations of it. I’m confused as to when things happened. When I saw that video you sent me it kind of jogged a little bit of a memory, then I saw some of the other ones. It sounds like I may have worked on three variations of that. The first one that was that prototype, the second one was the actual arcade game, and the third one was on the Bally Professional Arcade thing (on the little consumer box). When all those things happened, I can’t remember when they exactly happened.

\textcolor{interviewer}{Interviewer:} I imagine with this sort of stuff things kind of run parallel as well, you’re working on several projects at once.

\textcolor{interviewee}{Tom McHugh:} Well not really, no. I always worked on just one project. Up until you get to a certain point, you’re building a prototype – if you think of it that way – then when you’re done with the prototype you’re actually moving on. Then you may come back and do something based on feedback to build the final one but there’s not a whole lot more you’re gonna do there. It’s really only one project at a time, it’s not a multi-tasking thing.

\textcolor{interviewer}{Interviewer:} Can you tell me a bit about working on that game though? Were there any particular challenges?

\textcolor{interviewee}{Tom McHugh:} Not really. The biggest challenge, I think, was making the mazes, deciding what worked and what didn’t work. You’d come up with a mapping of a maze- you know, that’s really what decides those games is the mapping of a bunch of different mazes – you come up with one, try it out, see how it works, and if it makes sense (this is after all the other things were working).

You have the monsters running around – the wizard and the warlock doing their thing – then you just played around with mazes and make the whole thing fit together into a cohesive game that people would be interested in. For instance that business of what comes out when, when does it move faster, things like that.

\textcolor{interviewer}{Interviewer:} Was it difficult? Because the idea of the game is turning off the sprites when you can’t see them but you can potentially have all the sprites in one scanline. Did you have to make sure that it could handle that if that circumstance actually occurred or were you expecting it to not happen (have all the sprites on one thing)?

\textcolor{interviewee}{Tom McHugh:} No, no. That wasn’t a problem, I don’t think. Remember the thing that’s important there is the raster scan. So if they’re on one line at any point- I think the whole thing was basically randomly driven so every corner you come to is open three ways. The way you went was random. If it’s not open, of course, you can’t go that way.

\textcolor{interviewer}{Interviewer:} Putting this in a maze, were you working on this before Pac-man came out?

\textcolor{interviewee}{Tom McHugh:} Yes.

\textcolor{interviewer}{Interviewer:} So just a lucky coincidence that mazes were kind of in vogue at the time? [Laughs]

\textcolor{interviewee}{Tom McHugh:} Yeah, yeah. In fact I did a game called Amazing Maze, I think. The idea was there was an opening at the top and an opening at the bottom. You had to find your way through it in a certain amount of time or something like that, and I got so good at looking at mazes I could look at it and see the solution for it. The thing that created the maze was all random. 

It was kind of funny because I had done it over and over and over and over and over, so many times. I don’t think that game was very popular. [Laughs] I thought it was kind of cool, but after being with it for three months it got to be very easy to do. I think it was one of those games that was just too hard. No traction there.

[Side Note: Jamie Fenton also claims to have been the sole architect behind Amazing Maze. However, it is possible that McHugh might be thinking about the consumer version (which includes a Tic-Tac-Toe game), though that version is credited to Bill Jahnke.]

\textcolor{interviewer}{Interviewer:} How did you come up with those sort of psuedo-random procedures? Were there certain equations you were using? I’ve heard a lot of interesting ways that people kind of accomplished that. How did you do it?

\textcolor{interviewee}{Tom McHugh:} Yeah, I just picked a psuedo-random generator of some sort. I mean now what you’d do is go on the internet and look, but I had the book and I’d dive in the book and see what it said as to how you create something like that.

\textcolor{interviewer}{Interviewer:} I’ve heard of people using Fibonacci sequences and recursive lists. Lots of interesting ways in 4Kb that they’re able to make random numbers.

\textcolor{interviewee}{Tom McHugh:} Yeah. I don’t remember which one I picked.

\textcolor{interviewer}{Interviewer:} Another one of the cool things in Wizard of Wor is your next life comes out of the little cage at the bottom of the screen. Was that just a Dave idea or where did that come along?

\textcolor{interviewee}{Tom McHugh:} Oh yeah. Typically that’s the way games worked, you had some number of attempts to do something, that’s all that was. It was just standard gameplay.

\textcolor{interviewer}{Interviewer:} The idea of the life being represented on screen and then it becomes your character when it walks out of the thing.

\textcolor{interviewee}{Tom McHugh:} Yeah, that was probably a Dave Nutting thing.

\textcolor{interviewer}{Interviewer:} Was it always meant to be a wizard? You had Gun Fight where they have actual guns, then in this they have this kind of pest cone that they use for shooting.

\textcolor{interviewee}{Tom McHugh:} Well Dave was always playing with different inputs. One time he did one, I remember, it was with these little buttons called piezoelectric buttons. It’s a little button, it looks like a stack of maybe five quarters. It doesn’t move, but you’re touching it made it register that you’re trying to press it. In comparison to the older buttons that you push in (mechanical, more or less) I think he was looking for something that would not be as mechanical and fail at some point in the future.

He would come up with a lot of those things and typically I wouldn’t even see that. I would have something else that would be a lot more primitive but it would be similar to what would end up on the actual game itself. That’s how that part of it went. I would be working on software and he’d be working on hardware. 

He’d say, “I want to do this and we’re going to have this and this for our inputs.” You may have a joystick or a button or a combination of it or something. He knew how the inputs were going to interact with the game and he just had to build them.

\textcolor{interviewer}{Interviewer:} So that piezoelectric thing the idea was that you just touch it and then it registers?

\textcolor{interviewee}{Tom McHugh:} Yes.

\textcolor{interviewer}{Interviewer:} When you actually put stuff out did he build cabinets or was he mainly with circuitry?

\textcolor{interviewee}{Tom McHugh:} I think he build the cabinets too. I remember he needed something from a CRT, just the front part of it, so he ended up somehow or other busting the back end off of it. What I mean by the backend is a CRT has this space on it and it kind of tapers down to the back. All this taper stuff was above it too, so he busted all that stuff off of it because he wanted to use the front to do something. So I think part of what he did was actually build the cabinet itself.

\textcolor{interviewer}{Interviewer:} Obviously you were mainly on the software, but did you ever have to take into account what could be done in terms of the construction of the cabinet? Did you ever have any ideas-

\textcolor{interviewee}{Tom McHugh:} No. Absolutely not. Had nothing to do with that.

\textcolor{interviewer}{Interviewer:} The Wizard of Wor port for the Astrocade, the Bally console, that was an Action Graphics thing right?

\textcolor{interviewee}{Tom McHugh:} No, I don’t think so. No.

\textcolor{interviewer}{Interviewer:} So when you were making those games then you were just using the ICE boxes? It was kind of the same idea because it was the same processor?

\textcolor{interviewee}{Tom McHugh:} Yeah, yes. Remember, this Bally Professional Arcade thing, the way it worked was you had this thing you put on your TV set. It was basically a metal box that went into the antenna input on the TV then you could switch it with the TV one way or the other way (switch it to the TV or you could switch it to this arcade thing). What Jeff ended up doing was he did a direct drive to a monitor for me to use. I didn’t use the same technique and I’m not even sure how that worked.

\textcolor{interviewer}{Interviewer:} But he literally just wired it in? He just stripped some wires on got it in there?

\textcolor{interviewee}{Tom McHugh:} Yes.

\textcolor{interviewer}{Interviewer:} Those sorts of connections, those little forks are so difficult to get on your TV! [Laughs]

\textcolor{interviewee}{Tom McHugh:} Yeah, but Jeff knew what he was doing. He was very good at hardware. You asked me what I thought about Jeff and Dave: I thought both of them were really good at what they did.

\textcolor{interviewer}{Interviewer:} And certainly you were as well because you had to work within a paltry amount of memory and slow processors. Why I’ve really liked looking into this company particularly is because the things you had to do to make those games were just mind boggling in their time! Very interesting. In terms of arcade stuff was there anything after Wizard of Wor then?

\textcolor{interviewee}{Tom McHugh:} Oh, well like I said there may have been several variations with Wizard of Wor and there were things between that. […] We lived in this one place and I saved all the stuff, the folders with all the layouts and everything, and the final floppy disks. I had this big box of stuff, then we moved to another place, moved the box, then moved to another place, then we were gonna move again, and I said “You know, I’m gonna throw this box away” because I never look at it. Afterwards I thought, “You know, I shouldn’t have thrown that away at least for the perspective that I don’t even remember the games that I worked on.”

There was a lot of them. A lot of them didn’t do very well. One of them that I remember was called Clowns. It had three rows of balloons at the top of the screen and this guy that you bounced up to the top of the screen using like a springboard or something. Everytime you hit one of the balloons it made you pop up. The idea was you kept hitting the balloons until the balloons were all gone, that was a good thing, but if you didn’t keep hitting the balloons you would end up falling, that was bad thing. So there was a lot of those games that I kind of remember because it was such a silly thing, but I don’t think it did anything.

I remember seeing one of the- Again this was probably one of those that came over from Japan. It was kind of a similar game but the play was similar to what they had. That was always the intent. You’d take the concept that you bought from somebody else and if you can you make it better. At least you make it different because it’s being done with different hardware.

\textcolor{interviewer}{Interviewer:} Tell me a bit about the experience at Action Graphics then. Were they based in Wisconsin or were you doing remote stuff there as well?

\textcolor{interviewee}{Tom McHugh:} Well I was doing remote stuff for them but they were I think based in South (or Noth) Barrington which is one of the suburbs of Chicago. They weren’t that far from Arlington Heights. It may have been twenty miles or so, it wasn’t that far.
 
\textcolor{interviewer}{Interviewer:} Right. What’s some of the stuff that you worked on for them?

\textcolor{interviewee}{Tom McHugh:} Oh, I don’t even remember. I remember I did a Jeopardy game, I think [Family Feud]. I did another game that had to do with the Summer Olympics [The Activision Decathalon]. I think the games were mostly for the Colecovision. The Colecovision I think was based on a TI chip (which of course is no more, [Laughs] among other things).

\textcolor{interviewer}{Interviewer:} You never did Atari or something like that, right?

\textcolor{interviewee}{Tom McHugh:} Oh, Atari no, but with Action Graphics I ended up with a bunch of Apple hardware. That gave me the Family Feud game. I don’t know if that actually ever got finished, might have been one of those things they were going bankrupt and I had a bunch of hardware that belonged to them. That was among them, there might have been another one, it might have been an HP development system. That might have been for the Colecovision.

\textcolor{interviewer}{Interviewer:} Was the arrangement with Action Graphics pretty much the same as with Dave Nutting? You just come in every so often and get updates on everything?

\textcolor{interviewee}{Tom McHugh:} Yes, same thing. Same kind of financial arrangement and same kind of operating.

\textcolor{interviewer}{Interviewer:} Bob Ogden as a boss, how was he different than Dave?

\textcolor{interviewee}{Tom McHugh:} Well Dave was an industrial engineer and he wasn’t like the manager. He was kind of the owner or something like that. I don’t know if that’s a good distinction or not, but Bob was kind of just like “Manager”. Part of what he would do is he would go out and chase possible customers and try to tie them together. 

He had a guy that worked for him (can’t remember that guy’s name, he was like the chief engineer), Bob was the manager and he was the technical guy. They had a bunch of people in their offices there – I was there a couple times – but they were doing different kinds of things and they were working for other people too.

\textcolor{interviewer}{Interviewer:} Was it a bigger company than Dave Nutting Associates?

\textcolor{interviewee}{Tom McHugh:} No. Smaller. Dave Nutting had a lot of room there so apparently he might have had a number of people. This other one was maybe half that size. Again, that’s just me looking at it second hand and third hand.

\textcolor{interviewer}{Interviewer:} Right. After you got out of games then, you said you did some engineering work. Where did you go?

\textcolor{interviewee}{Tom McHugh:} I went to a company called Giddings \& Lewis in Fond Du Lac, Wisconsin. They make machine tools and I was in several of their groups there (engineering type of things). They were building a small processor. The first thing they hired me for, they had these really big machine tools that they started out with, kind of cost a million dollars.

They were run with mylar tape, so they’re kind of similar to paper tape that went into these machines for the first game that we did, the Gun Fight game. Same kind of things except the tapes were mylar because we had to use them over and over again. What they wanted someone to do was to come in and basically put DOS on one of their machine tools so they could just pop all these floppy disks in and get rid of the whole mylar thing. I did that, I made them DOS on a machine tool. Of course at that time DOS was completely defined so it wasn’t that hard.

\textcolor{interviewer}{Interviewer:} In terms of later stuff, what was your software specialty? Generally what were the tasks that you were doing day to day?

\textcolor{interviewee}{Tom McHugh:} Basically just “the programmer”. You want to do something, I’ll do it for you. One of the things I always did though was try to make it make sense. You ask me to do something dumb, I may change it. The designing part, I didn’t get into designing per say until later, other than the Wizard of Wor game I had a lot of input on that. How things move and how they work, a lot of that is just common sense. The whole concept of having both the player and the sprites (as they called them) moving both horizontally and vertically. Originally I think it was just all horizontal. I said, “It makes sense to do it horizontally and vertically so you can have vertical kinds of things happen too.” There’s a lot of input that I had on that.

\textcolor{interviewer}{Interviewer:} I forgot one thing. You said before you were working on a Wizard of Wor sequel and that kind of got canned because of Dave’s ideas. Was that specifically meant to be a Wizard of Wor sequel or were you just kind of taking ideas and building on that?

\textcolor{interviewee}{Tom McHugh:} I don’t know. I really don’t know.

\textcolor{interviewer}{Interviewer:} You said something about a space walk before.

[Side Note: After discussing with McHugh, this was not the game “Space Walk” released by Midway, which may have been a different DNA production.]

\textcolor{interviewee}{Tom McHugh:} Yeah, Dave was trying to tell me what he wanted to do and I said, “Yeah, I can do that but it doesn’t seem to make sense.” The actual Wizard of Wor game had the ability to go into different tunnels. This was just all freeform. There was a variation, one of the Wizard of Wor layouts was called “The Pit”. It had nothing in it but everything was still horizontal and vertical within that. What Dave wanted to do was completely freeform everything. I said, “Yeah, you can do that but it’s very hard then to decide where you’re going to move things.”

If you want to go horizontally or vertically, that’s relatively simple. If you’re heading in one direction, let’s say you’re heading right, on a horizontal you’re just incrementing one thing. Same thing for the vertical, you’re incrementing one thing, the X-Y kind of thing. If you want to do it everywhere, then what are you really doing? Really what you need is some kind of layout.

Wizard of Wor, the best way you could play it was defensively. Find a good position to sit, you could just sit there and shoot there. You could go hunting a little bit after that but you could just hunt and sit, hunt and sit. [In] this there was no place to sit, no place to hide. That became difficult. I kept trying to tell him that. [Laughs]

I remember the last time we got together on it before he canceled it we were living in Southwest Wisconsin, he said “I want to drive up there”. There was this big snowstorm, I said “Dave, do you really want to do it?” He said, “Yeah, I’ve got this Blazer. I can go anywhere with that.” And he did. [Laughs] He got there.

We sat down, we looked at it because he basically wanted face to face with it. He said, “This is what you’ve done so far.”, “What else can be done?”, “What can’t be done?” That was how that happened. After that was over and done with, after we had that sit down he decided basically to just cancel it because I didn’t know what the next step was.

\textcolor{interviewer}{Interviewer:} I imagine he didn’t go up to your place that often.

\textcolor{interviewee}{Tom McHugh:} That was the only time. It’s just funny that was during the snowstorm. We were way out in the boonies. I’m just thinking, “Do you want to do this?” He said, “I can make it.”

\textcolor{interviewer}{Interviewer:} Just to probe slightly more on that, he wasn’t trying to do the Wizard of Wor thing with the disappearing enemies because I don’t know how that can work if you just move everywhere all over the screen. Was that the idea?

\textcolor{interviewee}{Tom McHugh:} I don’t know. I never really got a good idea of where he was going with this. His basic initial instructions to me were, “Can you make it so that it goes anywhere and there’s no confining it horizontally and vertically?” I said, “Yeah, you could do that, but what’s the next step? What happens after that?” I kind of read into what he was saying as he was getting someone asking him, “Okay we have this Wizard of Wor thing. Let’s do the sequel to it.” He was trying to do something that would be a sequel but different from the first.

\textcolor{interviewer}{Interviewer:} Did you actually take the Wizard of Wor sprite and did that? You would be moving that around the screen or were you just moving basic shapes?

\textcolor{interviewee}{Tom McHugh:} Well first of all, when you say sprites, we never really did sprites. Sprites are really truly a different thing. For example, in the Coleco thing it had patterns and it had sprites, but you had maybe only two sprites or four sprites. A small number of actual sprites. These weren’t sprites. These were patterns that were being ripped after the raster. I think a sprite can be laid down any time you want to lay it down.

\textcolor{interviewer}{Interviewer:} Yeah. It ignores bitmap colors and things like that. So the stuff you were dealing with wasn’t really spirtes.

\textcolor{interviewee}{Tom McHugh:} Correct. They were not sprites.

\textcolor{interviewer}{Interviewer:} To summarize anything about your days at DNA? Was it a good experience in all? How do you feel about it?

\textcolor{interviewee}{Tom McHugh:} Oh yeah, I really enjoyed it. The only reason I ended up not doing it is because the whole industry went down. Well, I had family by that time. Had to deal with the kids [Laughs] and their appetites. I really enjoyed it. I liked Dave, I liked Jeff, I certainly did not dislike them by any means. Working there was ideal. It was a nice place to work while they were in the warehouse and after that working from home was...

My opinion now, has always been, working from home is a double-edged sword. I’ve never really been one that leans over one way or the other way. I enjoy working from home, I probably would have enjoyed working for them if they stayed in Milwaukee (wherever they went in Milwaukee). Just to have the ability to work remotely, that was nice, that was great. Nice people to work with.

\textcolor{interviewer}{Interviewer:} Thank you so much for your memories Tom! We’ve put you in the history books.
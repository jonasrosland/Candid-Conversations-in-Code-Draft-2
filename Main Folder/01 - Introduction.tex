\quad “Down to the metal” refers to a type of programming where the coder is but one step removed from the 1’s and 0’s of the logic gates which make up a circuit board. This abstract way of thinking about computers serves only the most dedicated of toolmakers today, but once upon a time it was the only way to make a box of chips do anything at all. This book then is dedicated to those who went “down to the metal” to create and revolutionize a force that we call video games.

This book presents a number of stories – some told for the first time – of game creators who began their careers in the 1970s. Some would only have the briefest of brushes with the industry before exiting, while others found vibrant professions which extended for decades from their first coding experiences. All of these creators were tutored in the halls of room-sized mainframe computing before becoming some of the first users of personal computers; often creating their own from scratch. They found home for their creative talents at companies like Atari, Exidy, and Dave Nutting Associates who together birthed the commercial industries which would come to define an oft-forgotten but wonderfully experimental age of video games.

Compared to a modern distributed architecture, these programmers could be said to have had it easy. With less metal on the line, it was entirely possible to learn and memorize every single thing that a simple 8-bit processor could achieve. However, what should not be lost is that the programmer of the 1970s also had to be a multi-disciplinarian. When “from scratch” is applied here, this would mean perhaps that a programmer had a compiler for their code to work with, and sometimes not even that. No pre-built computer, very little documentation, no built-in method for transferring the code, no tools, and often no time to get these done. It took a herculean effort to get from nothing to a viable product, but these enterprising programmers are the foundation upon which modern game development rests.

This book will hopefully help enlighten the struggles of these programmers and how specifically they created the means to make games in an open playing field. The rapid development of technology ends up obscuring this era of problem-solving so that those who managed to conquer the impossible remained obscure, even as they contributed to some of the most significant games of the decade. These stories may also provide inspiration for those who find struggles with their own projects –- technological or otherwise -- to find role-models in the largely positive and friendly people who I have reached out to over the past few years.

Despite many of these coders losing their job as the games industry entered periods of financial difficulty, none begrudge the time they spent creating these fun experiences. Readers may be surprised that many of them did not even enjoy games themselves, instead being drawn initially by the challenge of the task rather than what came out the other side. What remains clear though is that the game industry attracted a specific type of outsider who came away with a positive outlook on their experiences, looking fondly upon their pioneering. Most when contacted were astounded that anyone would take interest in things they accomplished 45 years prior, happy to relive their tumultuous experiences with some excellent detail to boot.

The interviews provided here are unequal in length and in some cases do not provide a complete overview of their career in games. Included with the mostly unedited transcripts are sidenotes which explain memories that might have been incorrect as well as historical detail to help place the development of these early microprocessor games in context. In addition, the conversations are mostly candid, meaning that they flow as a conversation would rather than as a script in a movie. This means that proper English may be sacrificed for accuracy and certain assumptions made on the part of the interviewer, who has read extensively for context prior to conducting these interviews. Additional resources can be found in the back of the book in the “Resources” section.

I hope that anyone who enjoys this work will feel free to tell others about it. Critique is also welcomed as this will be my first fully-published work. My contact email is historyhowweplay@gmail.com

Many thanks,

Ethan Johnson, June 2020
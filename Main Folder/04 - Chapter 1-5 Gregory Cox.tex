As an addition to this interview, also included is an edited transcript of the deposition Gregory Cox gave for the lawsuit of Bally v. Williams et. al from 1981. These memories are closer to the date and reveal many details otherwise obscured by the gap of over 45 years.

\noindent\makebox[\linewidth]{\rule{\paperwidth}{0.4pt}}

\textcolor{interviewer}{Examiner:} Give me your education after high school if any, what institutions you went to, what degree you got and what year you got it? 

\textcolor{interviewee}{Gregory Cox:} I have a Bachelors Degree in mathematics from San Jose State University in San Jose, California, granted in June of 1970. 

\textcolor{interviewer}{Examiner:} You joined Cyan Engineering in March of 1974, is that correct, sir? 

\textcolor{interviewee}{Gregory Cox:} Yes. 

\textcolor{interviewer}{Examiner:} Did you have any jobs prior to that time and subsequent to your getting a degree from San Jose State? 

\textcolor{interviewee}{Gregory Cox:} Yes, I did. 

\textcolor{interviewer}{Examiner:} Could you tell me what they were? 

\textcolor{interviewee}{Gregory Cox:} From January of 1967 through November of 1970 I was employed by Lockheed Missiles \& Space Company as a computer programmer. From November of 1970 to November of 1973 I was employed by Dalmo Victor, as a programmer . From November of 1973 to February of 1974 I was employed as a computer programmer by Ampex Corporation in Sunnyvale, California. I left Ampex at the end of February and joined Cyan immediately thereafter. 

\textcolor{interviewer}{Examiner:} What kind of computers were you programming when you were with Lockheed? 

\textcolor{interviewee}{Gregory Cox:} Various types of computers ranging from medium o moderately small sized computers up through very, very large scale computers. UNIAC 1108 and Sigma 5 and Sigma 7 were the principal computers that I programmed. 

\textcolor{interviewer}{Examiner:} How about at Dalmo Victor? 

\textcolor{interviewee}{Gregory Cox:} Programming there was on minicomputers and specialized processors that went into airborne military systems. These airborne processors fit somewhere between microprocessors and minicomputers. They tended to have the computational power of a minicomputer without the peripherals in memory capacity and general purpose features you find in the minicomputer. 

\textcolor{interviewer}{Examiner:} Now, while you were with Ampex, what kind of computers did you program?

\textcolor{interviewee}{Gregory Cox:} Minicomputers, Data General Eclipse type minicomputers. 

\textcolor{interviewer}{Examiner:} What was the first encounter or experience you had with microprocessors or microcomputers? 

\textcolor{interviewee}{Gregory Cox:} Oh, for a period of time prior to my employment at Cyan Engineering I had read trade journals and monitored the development in a very casual manner of microprocessor technology, and my direct involvement with microprocessors began with my employment at Cyan Engineering. 

\textcolor{interviewer}{Examiner:} Were you employed specifically for the purpose of programming microprocessor computers? 

\textcolor{interviewee}{Gregory Cox:} Yes. 

\textcolor{interviewer}{Examiner:} Prior to your employment did you have an interview or interviews with anybody?

\textcolor{interviewee}{Gregory Cox:} Steve Mayer, I believe I spoke with Larry Emmons also. 

\textcolor{interviewer}{Examiner:} Did you go there at their invitation? 

\textcolor{interviewee}{Gregory Cox:} No, I did not. 

\textcolor{interviewer}{Examiner:} How did the interviews come about? 

\textcolor{interviewee}{Gregory Cox:} I looked in the phone book for companies that did electronics work and went and knocked on their door. They happened to be in and willing to talk to me. 

\textcolor{interviewer}{Examiner:} So I take it during that period you were talking to several other companies as well or at least trying to? 

\textcolor{interviewee}{Gregory Cox:} Yes. I spoke with one other company in the Grass Valley area. 

\textcolor{interviewer}{Examiner:} Did you call and arrange the interview? Or did you just drive up to Grass Valley and- 

\textcolor{interviewee}{Gregory Cox:} I drove up to Grass Valley and while I was there, I was looking at real estate. And one of the real estate a gents referred me to several companies, and I used those referrals in conjunction with telephone listings and I called Cyan Engineering and they were willing to speak with me. 

\textcolor{interviewer}{Examiner:} Would it be proper to say that perhaps the determining factor in your mind was the location? You were interested in working in that part of the country? 

\textcolor{interviewee}{Gregory Cox:} Yes. 

\textcolor{interviewer}{Examiner:} Was there more than one interview, or did you make a deal that day? 

\textcolor{interviewee}{Gregory Cox:} There was that single interview and several subsequent telephone calls. To the best of my recollection that was the only interview that took place.

\textcolor{interviewer}{Examiner:} What was the time lapse between the Interview and when you started to work approximately? 

\textcolor{interviewee}{Gregory Cox:} About three weeks. 

\textcolor{interviewer}{Examiner:} At the time of the interview did Mr. Mayer tell you the business of Cyan Engineering? 

\textcolor{interviewee}{Gregory Cox:} He said they were involved in the design and fabrication of electromechanical and video games, which was the only detail he went into regarding their business. 

\textcolor{interviewer}{Examiner:} Well, when he said, "electromechanical games," did he give you any idea what kind of games? 

\textcolor{interviewee}{Gregory Cox:} No. The only information I was given on the job to the best of my recollection was that my duties would be applying microprocessors to games, and my specific duties would be for developing the software for those microprocessor applications. 

\textcolor{interviewer}{Examiner:} You would have no responsibility for developing hardware? 

\textcolor{interviewee}{Gregory Cox:} That’s correct. 

\textcolor{interviewer}{Examiner:} Were you told, sir, prior to your employment how they were going to use microprocessors, what they were going to do with them?

\textcolor{interviewee}{Gregory Cox:} As best I can recall, Steve and Larry mentioned that they intended to imbed microprocessors in various types of games to provide the control functions for those games. They were nonspecific as to the specific types of games.

\textcolor{interviewer}{Examiner:} Did they give you any idea what the inputs to these microprocessors would be?

\textcolor{interviewee}{Gregory Cox:} I don’t recall. We had a fairly lengthy technical discussion during the interview for a period of maybe one to two hours where Steve Mayer and Larry Emmons were trying to develop an understanding of my technical background and how it related to their task of implementing microprocessors. We talked about my experience with hardware interfaces to processors that I was familiar with. I suspect that they used the basis of those discussions to determine my qualifications regarding their specific applications. 

\textcolor{interviewer}{Examiner:} To the extent that you can recall prior to your joining Cyan Engineering, what was your understanding of the microprocessor or microcomputer? 

\textcolor{interviewee}{Gregory Cox:} Based upon information I read, early versions of microprocessors were basically expanded versions of calculator chips which provided a basic computer capability with very restricted and limited general purpose applications which would be programmed to provide numerical control sequences and perform limited analytical type functions That upon the initial applications of these types of devices, their potential was accomplished and a divergence in terms of technology took place that it was a conscious development of a general purpose microprocessor as distinguished frorm glorified calculator chips.

These new generation microprocessors were just becoming available in initial quantities at the time I was employed by Cyan Engineering. That there was a great deal of published information regarding the developments of various semi-conductor companies in this regard but actually, microprocessor hardware was still quite limited at that point. 

\textcolor{interviewer}{Examiner:} What were the dates of your employment with Cyan 11 Engineering? 

\textcolor{interviewee}{Gregory Cox:} I began work for Cyan Engineering on March 4, 1974. Terminated my employment with them on August 16, 1974. 

\textcolor{interviewer}{Examiner:} Now, at the time you joined Cyan Engineering and were given the assignment of preparing the software for the El Toro pinball, had any work on the software ware been attempted prior to your joining the company? 

\textcolor{interviewee}{Gregory Cox:} I am not certain. I remember Steve Mayer having some segments of a program which he had written and he asked me to review.  I don't know if they related to a pinball game development or another application.

\textcolor{interviewer}{Examiner:} At the time you joined Cyan Engineering, sir, had any hardware design been done on the El Toro game, if you know? 

\textcolor{interviewee}{Gregory Cox:} I am not certain, memory is not very clear on the subject.  I believe that Steve Mayer at least had done some preliminary hardware design prior to my employment within the company. 

\textcolor{interviewer}{Examiner:} Did Mr. Mayer sit with you and tell you what he wanted you to do? 

\textcolor{interviewee}{Gregory Cox:} Yes.

\textcolor{interviewer}{Examiner:} Did he explain the pinball game to you? 

\textcolor{interviewee}{Gregory Cox:} Yes. 

\textcolor{interviewer}{Examiner:} Was it necessary for him to do that? Had you ever encountered a pinball game before? 

\textcolor{interviewee}{Gregory Cox:} I had played pinball machines. That is the extent of my knowledge on them at the time I joined the company. 

\textcolor{interviewer}{Examiner:} Did Mr. Mayer seem to know what he was talking about to your understanding? 

\textcolor{interviewee}{Gregory Cox:} Relative to the subject of developing a microprocessor control system for a pinball machine, he seemed to be knowledgeable on what that task entailed, how to go about it. 

\textcolor{interviewer}{Examiner:} And did he provide you with instructions, sir, as to what you were to do? 

\textcolor{interviewee}{Gregory Cox:} Of a general nature. During the early phases of the program, there were continued discussions and design refinements based on discussions between myself and Steve Mayer relating to hardware and software implementation. The instructions were pretty much limited to developing a software system for the El Toro game. They were very general in nature. And we worked rather closely together in developing the basic input/output structure and control sequence structure of the hardware at an architectural level from which I proceeded to develop the software. 

\textcolor{interviewer}{Examiner:} What happened between your point of initial employment and this several weeks before you got these general instructions? 

\textcolor{interviewee}{Gregory Cox:} I was given a set of manuals for the 4004 microprocessor, for the INTELLEC MCS-4 system. I was given a programmer's guide. I was given instructions to become familiar with programming techniques, how to program them. I believe we took delivery of the INTELLEC MCS-4 system very shortly after I began work there. And one of my early tasks was to help bring that system to a running — up and running condition and to develop some support software for various aspects of the device . 

One of these that I recall was some software for enhancing the PROM programming capabilities of the machine. So during the first several weeks, I was becoming familiar with the INTELLEC system and its use. 

\textcolor{interviewer}{Examiner:} Did you find that a difficult task? 

\textcolor{interviewee}{Gregory Cox:} No, I did not. 

\textcolor{interviewer}{Examiner:} Were you able to complete those tasks, becoming familiar with the INTELLEC 4 and providing this support software? 

\textcolor{interviewee}{Gregory Cox:} Yes. There was a phone call that took place between myself and one of the applications engineers at Intel regarding software programming of the Intellec system. I recall a visit to the Cyan facility by either an applications engineer or a marketing person to see how we were getting along in using the Intellec 4. 

\textcolor{interviewer}{Examiner:} In your initial phase, your first several weeks with Cyan, sir, did you form any impression as to Mr. Mayer's technical abilities? Could you tell me what that Impression was? 

\textcolor{interviewee}{Gregory Cox:} In general, I thought he had a very excellent grasp of the technology he was dealing with, that he was a very innovative and creative electronics engineer. 

\textcolor{interviewer}{Examiner:} Did you form an impression as to whether or not he understood what microprocessors were or microcomputers and how they worked? 

\textcolor{interviewee}{Gregory Cox:} He showed more knowledge than I had, yes. 

\textcolor{interviewer}{Examiner:} Did there ever come a time when you formed any impression about his understanding of microprocessors and how they worked, sir? 

\textcolor{interviewee}{Gregory Cox:} Yes. By the time we had completed this project, I felt he had a good grasp as to the applications of microprocessors, yes. 

\textcolor{interviewer}{Examiner:} How did you gain your familiarity to whatever degree you had it of the hardware design? 

\textcolor{interviewee}{Gregory Cox:} Discussions with Steve Mayer and review of the designs and schematics for the electronics. 

\textcolor{interviewer}{Examiner:} When you say "review of the schematics," you mean actually studying the schematics yourself? 

\textcolor{interviewee}{Gregory Cox:} Yes, with Steve Mayer’s assistance. 

\textcolor{interviewer}{Examiner:} Anybody else?

\textcolor{interviewee}{Gregory Cox:} I don’t recall specifically, but it would not be improbable to have had conversations with Mike Rodgers or Ed Schleeter or Ron Milner regarding specific components and their operations or specific circuits and their operations.

\textcolor{interviewer}{Examiner:} These gentlemen that you just named. These were all employees of Cyan Engineering, were they not? 

\textcolor{interviewee}{Gregory Cox:} Yes.

\textcolor{interviewer}{Examiner:} In the course of writing the software for the El Toro project, did you ever receive any help from Mr. Mayer or anyone else at Cyan Engineering? 

\textcolor{interviewee}{Gregory Cox:} I was solely responsible for the design and implementation of the software. During the testing phase where we were debugging 1 the software and evaluating its performance on the pinball machine, the El Toro pinball machine, I had Steve Mayer's. assistance in operating the hardware and the electronics to perform such testing. 

\textcolor{interviewer}{Examiner:} How was the software for the El Toro game first tested? 

\textcolor{interviewee}{Gregory Cox:} Various portions of the software were tested independently. The program was written in certain logic loops or logic functions were tested individually. For instance, controlling the LED displays and these individual functions would have been tested one at a time. Most of these could be done without a significant amount of hardware involvement. 

In other words, implementing the hardware and the software, putting them together in the microprocessor, interpreting the microprocessor to the electronics in the El Toro game and triggering switches with finger or rolling the ball over. Testing was of that nature. 





\textcolor{interviewer}{Examiner:} Were you employed by Cyan Engineering during May of 18 1974? 

\textcolor{interviewee}{Gregory Cox:} Yes. 

\textcolor{interviewer}{Examiner:} Do you recall an open house that was held at Cyan Engineering in May of 1974? An event which took place during which Atari employees and their families visited the Cyan facility for a tour and were then taken to a park for a picnic.

\textcolor{interviewee}{Gregory Cox:} Yes. 

\textcolor{interviewer}{Examiner:} Do you recall an El Toro pinball machine being present at the Cyan Engineering open house? 

\textcolor{interviewee}{Gregory Cox:} Yes, I do.

\textcolor{interviewer}{Examiner:} The El Toro pinball machine that was present at the Cyan Engineering open house was controlled by a microprocessor, was it not? 

\textcolor{interviewee}{Gregory Cox:} Yes, it was. 

\textcolor{interviewer}{Examiner:} The El Toro pinball machine was played at the Cyan Engineering open house, wasn't it?

\textcolor{interviewee}{Gregory Cox:} To the best of my recollection, yes.

\textcolor{interviewer}{Examiner:} Did you observe anyone playing the El Toro pinball machine at the open house? 

\textcolor{interviewee}{Gregory Cox:} I don’t recall a specific event of observing any individual playing the El Toro game. 

\textcolor{interviewer}{Examiner:} Who else was present at the open house? 

\textcolor{interviewee}{Gregory Cox:} The Cyan Engineering employees, members of their family, a group of employees from the Atari facility in Los Gatos, and members of the Atari employees' families. 

\textcolor{interviewer}{Examiner:} Were there any children present at the open house? 

\textcolor{interviewee}{Gregory Cox:} I believe there were a few children present who were the members of the Cyan Engineering families.

\textcolor{interviewer}{Examiner:} Do you know the names of each person that was present? 

\textcolor{interviewee}{Gregory Cox:} No, I do not.

\textcolor{interviewer}{Examiner:} Do you remember the names of anyone who was present? 

\textcolor{interviewee}{Gregory Cox:} Yes, I do. 

\textcolor{interviewer}{Examiner:} Would you please give me the names of the people hat you remember who were present at the Cyan Engineering open 5 house?

\textcolor{interviewee}{Gregory Cox:} Larry Emmons, Steve Mayer, Ron Milner, Mike Rodgers, Jodie Sperry. Those are the only names I remember specifically.

\textcolor{interviewer}{Examiner:} Do you remember the names of anyone from the Atari facility in Los Gatos that was there? 

\textcolor{interviewee}{Gregory Cox:} No, I do not. 

\textcolor{interviewer}{Examiner:} Do you remember the names of anyone else who was there who may not have been from the Atari facility in Los Gatos or Cyan Engineering?

\textcolor{interviewee}{Gregory Cox:} Steve Mayer's wife was there. I do not remember her first name. Ron Milner's wife was also there. Those are the only other individuals I remember. 

\textcolor{interviewer}{Examiner:} Then there are some people that were there that you don't remember? 

\textcolor{interviewee}{Gregory Cox:} Yes. 

\textcolor{interviewer}{Examiner:} Who do you recall was not married at the time of the Cyan Engineering open house? 

\textcolor{interviewee}{Gregory Cox:} There was a student who was working as a technician for the summer. I don't recall his name, but I do recall that he was single. [Possibly Bob Walker]

\textcolor{interviewer}{Examiner:} As far as you know the El Toro pinball game played properly at the Cyan Engineering open house? 

\textcolor{interviewee}{Gregory Cox:} It played as it was designed to play, the design being the implementation of the microprocessor-based system, not the electromechanical system. 

\textcolor{interviewer}{Examiner:} Basically as far as the play of the game was to somebody sitting there, hitting the flippers and knocking the ball around, it was indistinguishable from the electromechanical version, is that correct? 

\textcolor{interviewee}{Gregory Cox:} There were some differences. 

\textcolor{interviewer}{Examiner:} Other than the sound and the appearance of the game being a little different because you did not have the relays clicking, would you agree that it would be indistinguishable?

\textcolor{interviewee}{Gregory Cox:} The relays were not present. The score display was using IDE's instead of a mechanical display. And there were interconnection wires emanating from the unit itself. Those were the differences between a normal electromechanical El Toro and the machine used at Cyan Engineering.

\textcolor{interviewer}{Examiner:} Do you specifically remember the steps that were in the El Toro machine software at each particular point in time during the development of the El Toro machine?

\textcolor{interviewee}{Gregory Cox:} The development of any software system involves a preliminary version of the program which is then tested and evaluated and corrected and modified during the development cycle of the software program. This sequence was used and implemented in the El Toro software development. At various times during the development of the software different versions of the software program were in use and "being evaluated within the machine. 

\textcolor{interviewer}{Examiner:} How do you distinguish or how can you tell one version of the software program from another? 

\textcolor{interviewee}{Gregory Cox:} During the development cycle of the El Toro program, certain changes were made to the basic operation of the software. I recall that at no time was there a major modification to the basic architecture of the software program. There were certain problems and difficulties encountered during the development of the software which were corrected. To the best of my recollection, this is the final version of the software program that was used in the El Toro machine and its microprocessor version during my employment with Cyan Engineering. 

While I was employed at Cyan Engineering, to my knowledge no other individual wrote or modified any of the software that was used in the El Toro game, that there were no other changes to this program other than what is documented in these two exhibits, and that this was the program used in the operation of the El Toro machine. 

\textcolor{interviewer}{Examiner:} The Cyan facilities were located in an old hospital building, were they not? 

\textcolor{interviewee}{Gregory Cox:} Yes. 

\textcolor{interviewer}{Examiner:} Was Cyan Engineering the only company or organization that had offices located in that building? 

\textcolor{interviewee}{Gregory Cox:} No. There were other companies who had facilities within that building also.

\textcolor{interviewer}{Examiner:} Do you recall the other companies that were located in the same building with Cyan Engineering in approximately May of 1974? 

\textcolor{interviewee}{Gregory Cox:} A I recall two companies. One was a division of Litton and the other was a company called Eigen Systems, I believe. 

\textcolor{interviewer}{Examiner:} Did the Cyan office have a lobby? 

\textcolor{interviewee}{Gregory Cox:} There was an area which could have been considered a lobby which contained Jodie Sperry's desk and Ed Schleeter's drafting table.

\textcolor{interviewer}{Examiner:} This area which I will refer to as the lobby then, do you recall a pinball machine ever being located in this lobby?

\textcolor{interviewee}{Gregory Cox:} No.

\textcolor{interviewer}{Examiner:} Would you draw a representation of the floor plan at the Cyan Facilities as you remember it? 

\textcolor{interviewee}{Gregory Cox:} Yes. That would be the floor plan as best I can recall it.

[Side Note: The Cyan Engineering floor plan drawn up by Gregory Cox in court.]

\textcolor{interviewer}{Examiner:} Is it correct that you do not remember one way or the other whether any express warning concerning confidentiality or secrecy was given to anyone at the Cyan Engineering open house? 

\textcolor{interviewee}{Gregory Cox:} Let me characterize the environment of secrecy and confidentiality and relate it to these issues. Basically, immediately upon being employed by Cyan Engineering I was advised that all work done there was confidential in nature and was not to be divulged outside of the immediate premises and the employees of the company. That attitude was prevalent in all the work that was done there. The drawings were kept in locked cabinets. There were burglar alarms that were activated during nonworking hours. All the information was very carefully guarded and protected.

There were periodic discussions about the requirements for confidentiality and secrecy. There were discussions prior to the open house dealing with the fact that there would be people from Atari there, that the Atari employees were privy to the information and to the work being done at Cyan Engineering and were aware of the policy of confidentiality that existed at a corporate level within Atari as well as Cyan Engineering. 

I can't state from specific recollection that there was an event during which it was specifically stated to all employees that we were having visitors and that we were to be reminded that all of this information was confidential. I guess an accurate characterization to my recollection would be that it was an ongoing policy which was periodically reinforced, which I recall being reinforced in the neighborhood of time of the tour with regard to the fact that there would be nonemployees there, the families of Atari members, and that for that reason we should be cautious about what information was disseminated. 

\textcolor{interviewer}{Examiner:} How was the objective of minimizing the amount of hardware used pursued in the design of the software? 

\textcolor{interviewee}{Gregory Cox:} Yes. By minimizing the number of instructions used for controlling the pinball machine to minimize the extent of memory required. By using an input /output structure which minimized the number of integrated circuits required to provide control functions from the microprocessor to the game, and to minimize the number of circuits required to provide inputs to the microprocessor. 

\textcolor{interviewer}{Examiner:} What basic hardware control functions were provided in the software? 

\textcolor{interviewee}{Gregory Cox:} Sensing the roll-over switches thumper bumper switches, sensing the out-hole, sensing the coin switch, providing controls for the chimes for the LED score display controls to activate solenoids to output a ball for play, to elect a hall from a bumper, to control the relating to the various switches that were activated when a switch was rolled over by a ball. 

There were probably others, but I would have to take some time and refer to the hardware and software documentation to be complete on all of those. 

\textcolor{interviewer}{Examiner:} Was the software divided into subroutines? 

\textcolor{interviewee}{Gregory Cox:} There were subroutines in the software, yes. 

\textcolor{interviewer}{Examiner:} What subroutines do you remember that were in the software? 

\textcolor{interviewee}{Gregory Cox:} There were various subroutines dealing with specific hardware control functions, transmitting of input /output commands to the — through the interface hardware and to the electromechanical portions of the El Toro game. There were subroutines for managing the accumulation of the score. There were subroutines for providing the LED output commands for control of the score display. 

\textcolor{interviewer}{Examiner:} Do you remember any examples of specific hardware control that you were referring to in your first subroutine group that you listed?

\textcolor{interviewee}{Gregory Cox:} Yes. One example would be the output of commands to the lights associated with the playfield switches. 

\textcolor{interviewer}{Examiner:} What other examples of specific hardware control do you recall? 

\textcolor{interviewee}{Gregory Cox:} Another example would be an input/output command for activation of solenoids associated with the thumper bumpers and the out-hole kicker. 

\textcolor{interviewer}{Examiner:} During the period that you were working on the El Toro software, were you working on any other tasks? 

\textcolor{interviewee}{Gregory Cox:} I believe that some of the work on the software for supporting the MCS-4 system was done during the time I was preparing the El Toro program. And it is also very likely that I was doing some work on another project right at the end of the El Toro software development phase. 

\textcolor{interviewer}{Examiner:} Tell me what that other project was, sir. 

\textcolor{interviewee}{Gregory Cox:} It was a little box with switches and LED's that would allow the operator to play blackjack, craps. 

\textcolor{interviewer}{Examiner:} And that was microprocessor controlled? 

\textcolor{interviewee}{Gregory Cox:} Yes. 

\textcolor{interviewer}{Examiner:} And it too was to use the MCS-4 system? 

\textcolor{interviewee}{Gregory Cox:} Yes . 

\textcolor{interviewer}{Examiner:} During this period, sir, was there only one MCS-4 system on Cyan's premises to your knowledge? 

\textcolor{interviewee}{Gregory Cox:} Yes. 

\textcolor{interviewer}{Examiner:} Was this project completed, the box with LED's so that you could play blackjack and craps?

\textcolor{interviewee}{Gregory Cox:} There were several other games. There were four games total. I don't remember what the other two were. 

\textcolor{interviewer}{Examiner:} When somebody was working on the game box did they have to disconnect the MCS-4 from the El Toro and hook it up to the game box? 

\textcolor{interviewee}{Gregory Cox:} If both games were to be played on the same instance , you would have to disconnect the MCS-4 from one and connect it to the other, yes. 

\textcolor{interviewer}{Examiner:} Was that a difficult task? 

\textcolor{interviewee}{Gregory Cox:} No.

Exmainer: How did you do that? How was that done? 

\textcolor{interviewee}{Gregory Cox:} There is what is called a ribbon cable, a thin flat cable which came out from the MCS-4 and was connected to the circuit card through a connector. All you had to do to connect the MCS-4 to a different machine was disconnect this particular ribbon cable and connect it to the other device. 

\textcolor{interviewer}{Examiner:} After the open house, sir, did you do any further work on the El Toro game? 

\textcolor{interviewee}{Gregory Cox:} I don’t recall.

\textcolor{interviewer}{Examiner:} If you didn't work on the El Toro game after the open house, what did you work on?

\textcolor{interviewee}{Gregory Cox:} The game box that was previously mentioned and another game. 

\textcolor{interviewer}{Examiner:} What was the other game, sir? 

\textcolor{interviewee}{Gregory Cox:} It was to be a flying game with a CRT providing a computer perspective of what a pilot might see in flight, a game controlled by a joystick.

\textcolor{interviewer}{Examiner:} And with a microprocessor control? 

\textcolor{interviewee}{Gregory Cox:} Yes. 

\textcolor{interviewer}{Examiner:} Your present recollection. Is that what you worked on until your employment ended at Cyan? 

\textcolor{interviewee}{Gregory Cox:} Yes. 

\textcolor{interviewer}{Examiner:} Why did your employment end at Cyan, sir? 

\textcolor{interviewee}{Gregory Cox:} The Atari Company was suffering some financial setbacks, as it was rumored to me, and all of the Cyan employees received a ten percent pay cut. I was having difficulty meeting my financial obligations prior to my pay cut and afterwards it became impossible. So I left the employ of the company to seek greener pastures that paid better.